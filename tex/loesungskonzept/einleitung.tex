\section{Einleitung}

Die Evaluation eines Lösungskonzeptes ist eine zentrale und kritische Phase
eines jeden Entwicklungsprojektes. Kritisch ist diese insbesondere aufgrund 
der Tatsache, dass funktionelle Entscheidungen getroffen werden aufgrund der
Rechercheergebnisse und der bisher gemachten Erfahrungen der einzelnen 
Projektteilnehmer. Diese Entscheidungsgrundlagen gilt es während dieser Phase
in möglichst effizienter und dennoch angemessener Weise zu festigen mit Hilfe
von gezielten Detailecherchen, praktischen Erfahrungen und theoretischen
Abschätzungen.

Dieses Dokument beschreibt den Prozess der durchlaufen wurde, welcher zur 
Festigung und Erweiterung der besagten Entscheidungsgrundlagen diente. Dieser
stellt eine starke Konkretisierung der ersten Rechercheergebnisse dar welche
direkt zur Zusammensetzung von Funktionen zu Lösungskonzepten führt. Die 
durchgeführte Evaluation brachte drei verschiedene Lösungskonzepte
hervor, welche nach folgender Strategie zustande kamen.

\begin{description}
	\item[Konzept 1] \emph{Das Motivationspferd} \\
		Das erste Konzept beschreibt die ideale Lösung nach welcher
		sich das ganze Team richtet. Dieses versucht möglichst allen
		Wunschvorstellungen gerecht zu werden ohne aus dem Rahmen
		des Vorstellbaren und Möglichen zu fallen.
	\item[Konzept 3] \emph{Der Kompromiss} \\
		Dieses Konzept dient als erste Alternative zum ersten Konzept
		und streicht von vornherein jene Funktionen welche 
		ausschliesslich der Erfüllung von Wunschanforderungen dienen
		und die voraussichtlich grosse Realisierungsaufwände 
		erfordern könnten.
	\item[Konzept 3] \emph{Das Fallnetz} \\
		Dieses Konzept beschreibt eine Lösung welche allen 
		Mindestanforderungen gerecht wird und dabei jeglicher 
		Mehraufwand zur Erfüllung von Wunschanforderungen
		extrahiert ist.
\end{description}

Nebst den verschiedenen Paradigmen untscheiden sich alle drei Konzepte
auch in der Art der Realisierung von Grundfunktionen bei denen die dazu
erforderliche Modularität gegeben ist. Beispielsweise ist es irrelevant
ob eine Entfernungsmessung per Ultraschall oder Laser erfolgt. Ob das
Gerät stationär ist oder sich bewegt beeinflusst aber sehr wohl viele 
andere Teilfunktionen in beträchtlicher Art und Weise.\\\\
\textbf{TODO struktur beschreiben}
