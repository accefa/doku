\subsection{Balltransport}
Die Bälle müssen die Maschinen mit einer konstanten Geschwindigkeit verlassen und das Ziel möglichst genau treffen. Um die Bälle zu beschleunigen, gibt es verschiedene Möglichkeiten.
\subsubsection{Luft}
Eine Möglichkeit um Bälle zu beschleunigen besteht darin, Druckluft zu verwenden. Die Bälle befinden sich in eine Röhre und werden direkt mit Druckluft in der Röhre beschleunigt. Mit einer präzisen Regulierung des Druckes, der Luft und des Volumenstromes kann die nötige Kraft erreicht werden. Dieses Methode könnte geeignet sein für unseres Projekt.
\subsubsection{Feder}
Der Benutzung der Federn erlaubt ein einfache Berechnung der Kräfte und Energien. Die grösste Schwierigkeit bei der Benutzung der Federn ist die korrekte Wiederpositionierung beim Nachladen, da die Federn sich verbiegen können. Damit das Problem des Nachladens wegfällt, könnten mehrere Abschussrampen gebaut werden.
\subsubsection{Drehräder}
Drehrädern sind die am meistens verwendete Beschleunigungsarten in Ballwerfsystemen. Bei der Benutzung von Drehrädern wird die tangentiale Geschwindigkeit der Drehräder auf den Objekt übertragen. Drehrädern können auf verschiedene Arten gebaut werden: zwei Gegenseitige drehende Räder oder ein Rad mit eine Führung.
\subsubsection{Zylinder (Pneumatisch)}
Nach eine vertiefende Recherche über die existierende pneumatische Zylindermodelle, haben wir entschlossen dass keiner unsere Anforderungen erfüllt. Zudem ergeben sich keine Vorteile gegenüber der Variante mit der Druckluft.  
\subsubsection{Zylinder (Elektromagnetisch)}
Nach eine vertiefende Recherche über die existierende elektromagnetische Zylindermodelle, haben wir entschlossen, dass keiner unsere Anforderungen erfüllt. Die Modelle die benutzbar wären, brauchen viel zu viel Energie während einer kurzen Zeitdauer.