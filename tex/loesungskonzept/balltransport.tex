\subsection{Balltransport}
Die Bälle müssen die "Maschine" verlassen und ins Korb hinein. Um die Bälle beschleunigen gibt es verschiedene Arten und Energieformen.
\subsubsection{Luft}
Die erste Möglichkeit um Bälle zu beschleunigen ist Druckluft. Die Bälle befinden sich in eine Röhre und werden direkt mit Druckluft gestossen. Mit eine präzise Regulierung des Druckes der Luft und des Volumenstromes kann die nötige Kraft erreicht werden, dieses Methode könnte sehr geeignet für unseres Projekt sein.
\subsubsection{Feder}
Der Benutzung der Federn erlaubt ein einfachere Berechnung der Kräfte und Energien. Die grösste Schwierigkeit bei der Benutzung der Federn ist die korrekte Wiederpositzionierung bei der Nachladung, da die Federn sich biegen können.
\subsubsection{Drehräder}
Drehrädern sind die am meistens erwändete Beschleunigungsarten in Ballwärfersysteme. Bei der Benutzung von Drehrädern wird die tangentiale Geschwindigkeit der drehende Rad auf den Objekt übertragen. Drehrädern können mit verschiedene Systeme zusammengesetzt werden: zwei Gegenseitige drehende Räder, ein Rad mit eine Führung oder ein rotierende Körper der das Ball stosst.
\subsubsection{Zylinder (Pneumatisch)}
Nach eine vertiefende Recherche über die existierende pneumatische Zylindermodelle, haben wir entschlossen dass keiner unsere Anforderungen erfüllt. 
\subsubsection{Zylinder (Elektromagnetisch)}
Nach eine vertiefende Recherche über die existierende elektromagnetische Zylindermodelle, haben wir entschlossen dass keiner unsere Anforderungen erfüllt. Die einige Modelle die benutzbar wären, brauchen hingegen viel zu viel Energie.