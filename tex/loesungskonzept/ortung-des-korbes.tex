\subsection{Ortung des Korbs}
Die Fragestellung lautet: "Wo ist der Standort des Korbes?". Um Standorte von Objekten herauszufinden gibt es mehrere Technologien. Die Erkennung hat folgende Herausforderungen:
\begin{itemize}
	\item Distanz 2 Meter
	\item Objekt ist schwarz
	\item Beleuchtung nicht konstant
\end{itemize}

\subsubsection{Optik}
Die Umsetzung mittels Optik erfordert auf der Seite der Software-Entwicklung erhöhten Aufwand im Gegensatz zu den anderen Umsetzungsmöglichkeiten.
\newline\newline
\textbf{Kamera mit Bildverarbeitung}\newline
Mittels einer Kamera kann ein Bild fotografiert werden, welches ausgewertet wird um den Ort des Korbes zu bestimmen. Geprüft wurden Auswertungsmethoden auf Basis von Java und Python. In beiden Sprachen gibt es mehrere Bibliotheken, welche es erlauben Bildverarbeitung durchzuführen. Im Bereich Java wurde ImageJ (\href{http://imagej.nih.gov/ij/}{http://imagej.nih.gov/ij/}) und in Phyton wurde OpenCV (\href{http://docs.opencv.org/}{http://docs.opencv.org/}) angeschaut. In beiden Programmiersprachen war es möglich Objekte zu erkennen. (siehe Anhang \ref{anhang-bildverarbeitung})\newline

\textbf{Pixy}\newline



\subsubsection{Ultraschall}

\subsubsection{Laser}

\subsubsection{Wärmebild}
Diese Umsetzung kommt aus mehreren Gründen nicht in Frage. Auf dem Markt gibt es keine Wärmebild Kameras, welche ins Budget passen. Andererseits wird sowohl der Korb wie auch die Wand dahinter dieselbe Wärme aufweisen.

\subsubsection{Radar}