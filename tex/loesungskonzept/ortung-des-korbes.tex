\subsection{Ortung des Korbs}
Die Fragestellung lautet: "Wo ist der Standort des Korbes?". Um Standorte von Objekte herauszufinden gibt es einige Technologien um welche es nachfolgend geht.

\subsubsection{Optik}

Pixy Section?

Kamera Modul für PI?

Bildverarbeitung
Mittels einer Kamera kann ein Bild fotografiert werden. Dieses Bild kann ausgewertet werden um den Ort des Korbes bestimmt werden. Geprüft wurden Auswertungsmethode auf Basis von Java und Python. In beiden Sprachen gibt es mehrere Bibliotheken, welche es erlauben Bildverarbeitung durchzuführen. Im Bereich Java wurde ImageJ (\href{http://imagej.nih.gov/ij/}{http://imagej.nih.gov/ij/}) und in Phyton wurde OpenCV  angeschaut. In beiden Programmiersprachen war es möglich Objekte zu erkennen. (siehe Anhang Bildverarbeitung)

\subsubsection{Ultraschall}

\subsubsection{Laser}

\subsubsection{Wärmebild}
Diese Umsetzung kommt aus mehreren Gründen nicht in Frage. Auf dem Markt gibt es keine Wärmebild Kameras, welche ins Budget passen. Andererseits wird sowohl der Korb wie auch die Wand dahinter dieselbe Wärme aufweisen.

\subsubsection{Radar}