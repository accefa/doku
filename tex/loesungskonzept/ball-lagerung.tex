\subsection{Ball-Lagerung}
Die Art der Balllagerung hängt zu einem grossen Teil vom gewählten Abwurfmechanismus ab. Der Hauptzweck der Lagerung besteht darin, die Bälle sicher dem Abwurfmechanismus zuzuführen. 

\subsubsection{Netz}
Eine Balllagerung in einem Netz ist nur sinnvoll, wenn alle Bälle zusammen in den Korb geworfen werden. Ansonsten ergibt sich kein Vorteil durch diese Art der Balllagerung. Bei den von uns erarbeiteten Konzepten macht diese Balllagerung keinen Sinn, da die Bälle geordnet dem Abwurfmechanismus zugeführt werden müssen.

\subsubsection{Magazin}
Bei dieser Variante werden die Bälle nacheinander dem Abwurfmechanismus zugeführt. Dies geschieht je nach gewählten Konzept mittels einer Feder oder einer Zahnstange. Der Vorteil dieser Variante besteht darin, dass die Bälle geordnet und geführt transportiert werden. Durch diese Vorteile ist diese Variante wohl die beste Lösung.

\subsubsection{Korb}
Ähnlich wie die Variante Netz. Diese Variante wird nicht weiterverfolgt, weil sich keine Vorteile gegenüber der Variante „Magazin“ ergeben.

\subsubsection{Rohr}
Ähnlich wie die Variante „Magazin“. Der Unterschied besteht darin, dass die Bälle nur mittels Schwerkraft dem Abwurfmechanismus zugeführt werden. Wenn die Schwerkraft ausreicht um die Bälle dem Abwurfmechanismus zuzuführen ist dies klar der einfachste Lösungsansatz.