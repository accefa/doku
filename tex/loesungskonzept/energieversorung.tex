\subsection{Energieversorgung}

Der autonome Ballwerfer muss mit Energie versorgt werden. Die Anforderungen beschränken die Art der Energieversorgung nicht. Da das Gewicht der Energieversorgung nicht dem Gesamtgewicht angerechnet wird, sollte sie möglichst einfach entnehmbar sein. In den folgenden Abschnitten werden verschiedene Arten einer möglichen Energiezuführung kurz erläutert.

\subsubsection{Akku}

Ein Akku als Stromlieferant ist vor allem bei einem fahrbaren Gerät Pflicht. Jedoch haben wir uns geeinigt dass unser Gerät nicht fahrbar ist und somit verliert diese Art der Energieversorgung an Relevanz. Ein Netzteil eignet sich besser, weil für mehr Leistung weniger Kosten anfallen. Gerade bei Elektromagneten oder starken Motoren gerät ein normaler Akku schnell an seine Grenzen.

\subsubsection{Drucktank}

Ein Drucktank eignet sich dafür ein mobiles Gerät zu Versorgen oder kurzfristig einen hohen Druck zur Verfügung zu stellen. Wie im oberen Abschnitt schon erwähnt haben wir uns gegen eine mobile Lösung entschieden. Jedoch kann ein Drucktank den nötigen Druck liefern um einen Ball direkt mit Luft abzuschiessen oder einen starken Pneumatikzylinder zu betreiben. Zudem kann der Drucktank verwendet werden um mögliche Schwankungen auf dem Luftdrucknetz auszugleichen.

\subsubsection{Dampf}

Die Zeit der Dampflokomotiven ist wohl definitiv vorbei. Diese Möglichkeit ist während des Ideenfindungsprozesses entstanden und wird nur der Vollständigkeit halber erwähnt.

\subsubsection{Netzteil}

Ein Netzteil liefert den nötigen Strom für eine Vielzahl von Komponenten und wird definitiv in unserem Gerät verwendet. Da die Kosten eines vorhandenen Netzteils nicht angerechnet werden, kann man hier genügend Leistung zu einem akzeptablen Budget bereitstellen. Das Netzteil wird das schwächste Glied in der Kette, dem trotz seiner Einfachheit genügend Beachtung geschenkt werden sollte. Den ohne das Netzteil läuft der Bordcomputer nicht und auch die restlichen elektrischen Komponenten versagen ihren Dienst.