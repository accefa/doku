\subsection{Positionierung}
In der Funktion Positionierung geht es darum wie die Maschine sich ausrichten soll.

\subsubsection{Fix}
Dies ist die einfachste Variante. Es werden keine zusätzliche Komponenten benötigt damit die Maschine sich in irgendeiner Form bewegt. Das setzt dann aber Voraus, dass bewegliche Achsen implementiert werden, damit die Maschine auf die Variabilität des Standortes des Korbes agieren kann.

\subsubsection{Springt in Korb}
Diese konkrete Umsetzung meint, dass die Maschine alle fünf Bälle packt und sich selbst mit den Bällen in den Korb transportiert bzw. springt. Diese Variante wird ausgeschlossen, da es uns gegenüber anderen möglichen Umsetzungen zu komplex erscheint.

\subsubsection{Fliegen}
Mit dieser Methode startet die Maschine ihre Propeller und transportiert die Bälle im Flug in den Korb. Auch diese Variante wird ausgeschlossen, da das Fliegen eine ganze Palette an Unbekannten mitbringt in welchem dem Team das Know-how fehlt. Beispielsweise ist ein stabiler Flug zu implementieren einfach gesagt als getan.

\subsubsection{Geradeaus fahren / Rollt}
Möglich ist es, dass die Maschine bis zur Grenze fährt um dem Korb näherzukommen. Auch dies wird ausgeschlossen, den der Aufwand dafür ist zu gross um nur 'etwas' näher zu sein. Es sollte sich lohnen mehr Aufwand in eine gute Abschussvorrichtung zu investieren. 
