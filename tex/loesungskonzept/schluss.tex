\section{Schluss}
Innerhalb von vier Wochen hat das Team 39 drei Grobkonzepte erarbeitet, welche auf den Anforderungen und den Rechercheergebnisse den Wochen zuvor basieren. Die Analyse der benötigten Funktionen des Gesamt-Systems war der Startschuss für das vorliegende Lösungskonzept. Die Ergebnisse daraus wurden in einem morphologischen Kasten festgehalten. Jeder hat für jede Funktion konkrete Umsetzungsmöglichkeiten genannt, welche nachfolgend entsprechend detaillierter angeschaut wurden. Einige waren wortwörtliche 'Hirngespinster' (z.B. beim Balltransport: Beamen) andere wiederum wurden in die Grobkonzepte eingearbeitet. \\
\\
Nachdem klar wurde wie man welche Funktion möglicherweise umsetzen kann, wurden Lösungsskizzen gezeichnet. So kamen einige interessante Ideen zusammen. Mit der Zeit und nach diversen Versuchen hat sich auch herauskristallisiert, dass es konzept-unabhängige Funktionen gibt. Diese Teilaufgaben werden in jedem Konzept gleich implementiert. Darunter fällt die 'Ortung des Korbes', der 'Bordcomputer' und die 'Kommunikation'. Aufwendige Versuche mit Laser-, Ultraschall- und Optikmodulen sowie Kommunikationstests wurden durchgeführt und lieferten wichtige Erkenntnisse. So war man überrascht, dass kostengünstige Laser- und Ultraschallapparaturen ungenau und störanfällig sind. Auf Seite der Mechanik hat man sich auf drei konkrete Lösungsideen fixiert und diese in unabhängigen Versuchen überprüft ob die gewählten Varianten im Grundsatz auch umsetzbar ist.\\
\\
Im letzten Teil wurden die drei Grobkonzepte nach einem einheitlichen Raster bewertet. Der Gewinner ist das Konzept 'Drehrad'. Wichtiger als die Bewertung selbst ist, dass das ganze Team an das Gewinner-Konzept glaubt und mit einem guten Bauchgefühl die nächsten Aufgaben angehen kann. Nach einer intensiven und spannenden Zeit können wir nun drei Grobkonzepte mit gutem Gewissen präsentieren.