\section{Kommunikation}
\label{anhang-kommunikation}
Der ganze Versuch basiert auf dem Tutorial von dreamyssoft.com \href{http://www.dreamsyssoft.com/blog/blog.php?/archives/6-Create-a-simple-REST-web-service-with-Python.html}{http://www.dreamsyssoft.com/blog/blog.php?/archives/6-Create-a-simple-REST-web-service-with-Python.html}

\subsection{Ziel}
Das Ziel des Versuches ist es über den TCP/IP Stack von einem externen Gerät Funktionen auf dem Raspberry PI aufzurufen.

\subsection{Versuch}
In einem ersten Schritt wurde auf dem Raspberry PI ein Python Webservice implementiert und gestartet. Die Wahl Python ist dadurch begründet, da mittels Python relativ einfach serielle Ports angesteuert werden können.\newline
\newline
Dazu haben wir auf dem Raspberry PI die Python-Library web.py installiert. Diese Library ermöglicht es einen Webservice in Python zu schreiben. Nachfolgend haben wir den Webservice auf dem PI angelegt und anschliessend gestartet.

\begin{lstlisting}
#!/usr/bin/env python
import web
import xml.etree.ElementTree as ET

tree = ET.parse('user_data.xml')
root = tree.getroot()

urls = (
	'/users', 'list_users',
	'/users/(.*)', 'get_user'
)

app = web.application(urls, globals())

class list_users:        
	def GET(self):
		output = 'users:[';
			for child in root:
				print 'child', child.tag, child.attrib
				output += str(child.attrib) + ','
				output += ']';
				return output

	class get_user:
	def GET(self, user):
		for child in root:
			if child.attrib['id'] == user:
			return str(child.attrib)

	if _name_ == "_main_":
		app.run()
\end{lstlisting}

Nun läuft der Webservice unter dem Port 8080 und kann von einem x-beliebigen Gerät im gleichen Netzwerk aufgerufen werden.
\subsection{Fazit}
Es ist problemlos möglich über TCP/IP auf das Raspberry PI zuzugreifen. Sogar Webservices lassen sich in Python implementieren, welche von irgendeinem Client aufgerufen werden können.

