\section{Bildverarbeitung}
Die Bildverarbeitung wurde auf Basis von Java und Python geprüft. In beiden Programmiersprachen gibt es viele Bibliotheken, welche Bildoperationen unterstützen. Es beginnt bei einfachen Funktionen wie das Zuschneiden oder Komprimieren von Bildern über Objekterkennung.

Java
In Java wurde die Bibliothek ImageJ \href{http://imagejdocu.tudor.lu/}{http://imagejdocu.tudor.lu/} angeschaut. ImageJ ist modular aufgebaut und lässt sowohl Nutzen als auch Erweitern. Zweiteres macht die Bibliothek Attraktiv, weil es bereits einige Erweiterungen im Netz gibt. Eine solche Erweiterung nennt sich FeatureFinder. Diese Erweiterung verlangt als Input das zu suchende Objekt als Bild und das Bild auf welchem das Objekt gefunden werden muss.

Eine zweite Variante ist auf einer definierten Linie auf dem Bild die Pixel auszulesen. Da der Korb dunkler ist als der Hintergrund kann man nach dem Muster des Korbes suchen.

Python


