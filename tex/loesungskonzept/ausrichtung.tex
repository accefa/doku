\subsection{Ausrichtung}
Um Zeit einzusparen und die Komplexität zu reduzieren, wird die Ballwurfmaschine in der Mitte des Spielfeldes positioniert. Aus diesem Grund muss das Gerät in horizontaler wie auch in vertikaler Richtung ausgerichtet werden können, nachdem der Korb positioniert wurde. 

\subsubsection{horizontale Ausrichtung}
Die horizontale Ausrichtung geschieht mit einem Schrittmotor. Dabei wird der Abwurfmechanismus so eingestellt, dass er in Richtung des Korbes zeigt. Um die Komplexität zu reduzieren, befindet sich die Ausgangsposition ganz links oder rechts aussen. So muss die Drehbewegung nur in eine Richtung ausgeführt werden.

\subsubsection{vertikale Ausrichtung}
Damit möglichst wenige Motoren verbaut werden, ist es sinnvoll, die horizontale und vertikale Ausrichtung Mechanisch zu koppeln. Dabei wird die Ballwurfmaschine in horizontaler Richtung mit einem Schrittmotor gedreht. Bei dieser Drehbewegung wird der Abwurfmechanismus über eine Kurvenscheibe geführt, welche den Abschusswinkel einstellt. Dies ist nötig, damit die Tennisbälle mit konstanter Kraft abgeschossen werden können.