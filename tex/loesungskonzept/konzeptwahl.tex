\section{Konzeptwahl}
Alle Konzepte werden nach einem einheitlichem Bewertungsraster bewertet. Jeder Punkt im Raster ist mit einer Gewichtung versehen. Diese Gewichtung ist ein Wert von 1 bis 3. Jeder Punkt wird selbst mit einem Wert von 0 bis 3 bewertet. Nachfolgend alle Punkte und die Bedeutung der Bewertung:\newline
\begin{itemize}
	\item Preis (Gewichtung 3)
	\subitem 0 Punkte - Preis liegt nicht im Budget
	\subitem 3 Punkte - Preis liegt im Budget
	
	\item Balltransport (Gewichtung 3)
	\subitem 0 Punkte - Kann 2 oder weniger Bälle transportieren
	\subitem 1 Punkt - Kann mind. 3 Bälle transportieren
	\subitem 2 Punkte - Kann mind. 4 Bälle transportieren
	\subitem 3 Punkte - Kann alle Bälle transportieren
	
	\item Robustheit (Gewichtung 3)
	\subitem 0 Punkte - Umgebungsbedingungen haben starken Einfluss
	\subitem 1 Punkte - Umgebungsbedingungen haben geringen Einfluss
	\subitem 3 Punkte - Umgebungsbedingungen haben keinen Einfluss
	
	\item Zeit (Gewichtung 2)
	\subitem 0 Punkte - Vorgang dauert länger als Anforderungen
	\subitem 1 Punkt - Vorgang dauert lange
	\subitem 2 Punkte - Vorgang dauert schnell
	\subitem 3 Punkte - Vorgang dauert sehr schnell
	
	\item Gewicht (Gewichtung 1)
	\subitem 0 Punkte - Mehr als 4 Kilogramm
	\subitem 1 Punkte - Maximal 4 Kilogramm
	\subitem 3 Punkte - Maximal 2 Kilogramm
	
	\item Umsetzbarkeit (Gewichtung 2)
	\subitem 0 Punkte - Viele Fragezeichen wie es umgesetzt wird
	\subitem 1 Punkte - Fragezeichen wie es umgesetzt wird vorhanden
	\subitem 3 Punkte - Keine Fragezeichen wie es umgesetzt wird
	
	\item Aufwand (Gewichtung 2)
	\subitem 0 Punkte - Nicht möglich in der gegeben Zeit umzusetzen
	\subitem 1 Punkte - Sehr aufwändig umzusetzen
	\subitem 3 Punkte - Aufwand hält sich im Rahmen	
	
	\item Anforderungen (Gewichtung 3)
	\subitem 0 Punkte - Erfüllt Pflichtanforderungen nicht
	\subitem 2 Punkte - Erfüllt Pflichtanforderungen
	\subitem 3 Punkte - Erfüllt Pflicht- und Wunschanforderungen
	
	\item Bauchgefühl (Gewichtung 3)
	\subitem 0 Punkte - Schlechtes Bauchgefühl
	\subitem 1 Punkte - Gemischtes Bauchgefühl
	\subitem 3 Punkte - Gutes Bauchgefühl
\end{itemize}
Nach dem Bewerten der einzelnen Konzepte werden die konkrete Punktgebung mit den Gewichtungen multipliziert. Alle Multiplikationen werden summiert.

\begin{table}[h!]
	\renewcommand{\arraystretch}{1.5}
	\centering
	\begin{tabular}{l l l l l}
		Bewertungspunkt & Gewichtung & Konzept Drehrad & Konzept Luftdruck & Konzept Propeller \\
		\hline
		Preis 			& 3 & 3 (9) & 3 (9) & 3 (9) \\
		Balltransport 	& 3 & 2 (6)	& 1 (3) & 1 (3) \\
		Robustheit 		& 3 & 3 (9) & 3 (9) & 3 (9) \\
		Zeit 			& 2 & 2 (4) & 0 (0) & 3 (6) \\
		Gewicht 		& 1 & 1 (1) & 3 (3) & 0 (0) \\
		Umsetzbarkeit 	& 2 & 1 (2) & 1 (2) & 0 (0) \\
		Aufwand 		& 2 & 3 (6) & 1 (2) & 1 (2) \\
		Anforderungen 	& 3 & 2 (6) & 2 (6) & 2 (6) \\
		Bauchgefühl 	& 3 & 3 (0) & 0 (0) & 1 (3) \\
		\hline
		Total 			&  	& 46 	& 34 	& 36 	\\
	\end{tabular}
	\caption{Bewertung}
	\label{tab:bewertung}
\end{table}

Das Konzept Drehrad hat nach der Bewertung die höchste Punktzahl erreicht und wird somit ausgewählt.