\section{Konzeptwahl}
Alle Konzepte werden nach einem einheitlichem Bewertungsraster bewertet. Jeder Punkt im Raster ist mit einer Gewichtung versehen. Diese Gewichtung ist ein Wert von 1-3. Jeder Punkt wird dann selbst mit einem Wert von 0-3 bewertet. Nachfolgend alle Punkte und die Bedeutung der Bewertung:\newline
\begin{itemize}
	\item Preis (Gewichtung 3)
	\subitem 0 Punkte - Preis liegt nicht im Budget
	\subitem 3 Punkte - Preis liegt im Budget
	
	\item Balltransport (Fähigkeit der Maschine die Bälle zu transportieren darunter zählt auch die Präzision) (Gewichtung 3)
	\subitem 0 Punkte - Kann 2 oder weniger Bälle transportieren
	\subitem 1 Punkt - Kann mind. 3 Bälle transportieren
	\subitem 2 Punkte - Kann mind. 4 Bälle transportieren
	\subitem 3 Punkte - Kann alle Bälle transportieren
	
	\item Robustheit (Robust gegenüber tangierende äussere Einflüsse wie Luftfeuchtigkeit oder Licht) (Gewichtung 3)
	\subitem 0 Punkte - Grosse Probleme vorhanden
	\subitem 1 Punkte - Braucht gewisse Voraussetzungen
	\subitem 3 Punkte - Es sollte robust sein
	
	\item Zeit (Gewichtung 2)
	\subitem 0 Punkte - Vorgang dauert länger als 40 Sekunden
	\subitem 1 Punkt - Vorgang dauert max. 40 Sekunden
	\subitem 2 Punkte - Vorgang dauert max. 30 Sekunden
	\subitem 3 Punkte - Vorgang dauert max. 20 Sekunden
	
	\item Gewicht (Gewichtung 1)
	\subitem 0 Punkte - Mehr als 4 Kilogramm
	\subitem 1 Punkte - Maximal 4 Kilogramm
	\subitem 3 Punkte - Maximal 2 Kilogramm
	
	\item Umsetzbarkeit (Gewichtung 2)
	\subitem 0 Punkte - Viele Fragezeichen wie es umgesetzt wird
	\subitem 1 Punkte - Fragezeichen wie es umgesetzt wird vorhanden
	\subitem 3 Punkte - Keine Fragezeichen wie es umgesetzt wird
	
	\item Aufwand (Gewichtung 2)
	\subitem 0 Punkte - Nicht möglich in der gegeben Zeit umzusetzen
	\subitem 1 Punkte - Sehr aufwändig umzusetzen
	\subitem 3 Punkte - Aufwand hält sich im Rahmen	
	
	\item Anforderungen (Gewichtung 3)
	\subitem 0 Punkte - Erfüllt Pflichtanforderungen nicht
	\subitem 2 Punkte - Erfüllt Pflichtanforderungen
	\subitem 3 Punkte - Erfüllt Pflicht- und Wunschanforderungen
	
	\item Bauchgefühl (Gewichtung 1)
	\subitem 0 Punkte - Schlechtes Bauchgefühl
	\subitem 1 Punkte - Gemischtes Bauchgefühl
	\subitem 3 Punkte - Gutes Bauchgefühl
\end{itemize}
Nach dem bewerten der einzelnen Konzepte werden die konkrete Punktgebung mit den Gewichtungen multipliziert. Alle Multiplikationen werden summiert.

\begin{table}[h!]
	\centering
	\begin{tabular}{l l l l}
		Kriterium & Gewichtung & Bewertung & Total \\
		\hline
		
		Preis & 3 & ? & ? \\
		Balltransport & 3 & ? & ? \\
		Robustheit & 3 & ? & ? \\
		Zeit & 2 & ? & ? \\
		Gewicht & 1 & ? & ? \\
		Umsetzbarkeit & 2 & ? & ? \\
		Aufwand & 2 & ? & ? \\
		Anforderungen & 3 & ? & ? \\
		Aufwand & 1 & ? & ? \\
		Total &  &  & ? \\
	\end{tabular}
	\caption{Bewertung}
	\label{tab:quelle}
\end{table}