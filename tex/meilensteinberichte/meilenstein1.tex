\begin{table}[!h]
	\renewcommand{\arraystretch}{1.5}
	\begin{tabularx}{\textwidth}{|X|X|}
		\hline
		\multicolumn{2}{|c|}{
			\parbox[0pt][3em][c]{0cm}{}
			{\large \textbf{Meilensteinbericht Gruppe 39}}
		} \\
		\hline
		\multicolumn{2}{|l|}{Projekt: Autonomer Ballwerfer} \\
		\hline
		\multicolumn{2}{|p{0.95\textwidth}|}{
			Teilnehmer:
			\begin{itemize}
				\setlength\itemsep{0em}
				\item Martin Vogel \textit{Dozent}
				\item Adriano Valsangiacomo
				\item Ervin Mazlagi\'c
				\item Christian Spycher
				\item Fabian Wüthrich
				\item Christian Schürch
				\item Alexander Suter
			\end{itemize}
		} \\
		\hline
		Datum: 10. Oktober 2014 & Meilenstein: Testat 1 \\
		\hline
		\multicolumn{2}{|p{0.95\textwidth}|}{
			Überblick: \newline
			$\boxtimes$ Ziele erreicht \newline 
			$\square$ Es gibt einige Fragezeichen \newline
			$\square$ Fortschritt blockiert
		} \\
		\hline
		\multicolumn{2}{|p{0.95\textwidth}|}{
			Inhalt: \newline
			\textbf{Anforderungen}
			\begin{itemize}
				\item Trefferquote von 20 - 80 \% in 2 Minuten ist zu wenig
				\item 2 Minuten werden als Mindestanforderung definiert
				\item 30 Sekunden als Wunschanforderung
				\item Trefferquote wird auf 100 \% erhöht
				\item Es soll definiert werden wie die Maschine ausgerichtet werden soll (z.B. Schablone)
				\item Oberfläche noch genauer spezifizieren
				\item Neigungswinkel von 0.3$^\circ$ auf 2$^\circ$ erhöhen
				\item Licht ist mit 100'000 lm zu hell
			\end{itemize}
			\textbf{Ausblick}
			\begin{itemize}
				\item Für nächsten Meilenstein drei Konzepte ausarbeiten
				\item Am Schluss eines davon ins Detail ausarbeiten
			\end{itemize}
			Gruppe 39 erhält das Testat wenn die oben genannten Punkte korrigiert worden sind.
		} \\
		\hline
	\end{tabularx}
\end{table}
