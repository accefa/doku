\begin{table}[!h]
	\renewcommand{\arraystretch}{1.5}
	\begin{tabularx}{\linewidth}{|X|X|}
		\hline
		\multicolumn{2}{|c|}{
			\parbox[0pt][3em][c]{0cm}{}
			{\large \textbf{Meilensteinbericht Gruppe 39}}
		} \\
		\hline
		\multicolumn{2}{|l|}{Projekt: Autonomer Ballwerfer} \\
		\hline
		\multicolumn{2}{|p{0.95\textwidth}|}{
			Teilnehmer:
			\begin{itemize}
				\setlength\itemsep{0em}
				\item Martin Vogel \textit{Dozent}
				\item Adriano Valsangiacomo
				\item Ervin Mazlagi\'c
				\item Christian Spycher
				\item Fabian Wüthrich
				\item Christian Schürch
				\item Alexander Suter
			\end{itemize}
		} \\
		\hline
		Datum: 12. Dezember 2014 & Meilenstein: Testat 3 \\
		\hline
		\multicolumn{2}{|p{0.95\textwidth}|}{
			Überblick: \newline
			$\boxtimes$ Ziele erreicht \newline 
			$\square$ Es gibt einige Fragezeichen \newline
			$\square$ Fortschritt blockiert
		} \\
		\hline
		\multicolumn{2}{|p{0.95\textwidth}|}{
			Inhalt: \newline
			\textbf{Korrekturen Testat 3}
			\begin{itemize}
				\item Mehr Bilder - morphologischer Kasten als Bild
				\item Ein Konzept hat einfach 46 Punkte. Warum? - Fazit fehlt
			\end{itemize}
			\textbf{Informatik}
			\begin{itemize}
				\item Objekterkennung eher nicht - da kompliziert
				\item Kontrast und Helligkeit während Vorbereitungszeit einstellen
				\item Weissabgleich vornehmen
			\end{itemize}
			\textbf{Elektrotechnik/Maschinenbau}
			\begin{itemize}
				\item Encoder für Regelung einbauen
				\item Wie wird Drehrad gelagert (Bruch an Achse)
				\item Zylinder nochmals überprüfen
			\end{itemize}
			Gruppe 39 erhält das Testat wenn die oben genannten Punkte korrigiert worden sind.
		} \\
		\hline
	\end{tabularx}
\end{table}