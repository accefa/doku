\begin{table}[!h]
	\renewcommand{\arraystretch}{1.5}
	\begin{tabularx}{\linewidth}{|X|X|}
		\hline
		\multicolumn{2}{|c|}{
			\parbox[0pt][3em][c]{0cm}{}
			{\large \textbf{Meilensteinbericht Gruppe 39}}
		} \\
		\hline
		\multicolumn{2}{|l|}{Projekt: Autonomer Ballwerfer} \\
		\hline
		\multicolumn{2}{|p{0.95\textwidth}|}{
			Teilnehmer:
			\begin{itemize}
				\setlength\itemsep{0em}
				\item Martin Vogel \textit{Dozent}
				\item Adriano Valsangiacomo
				\item Ervin Mazlagi\'c
				\item Christian Spycher
				\item Fabian Wüthrich
				\item Christian Schürch
				\item Alexander Suter
			\end{itemize}
		} \\
		\hline
		Datum: 7. November 2014 & Meilenstein: Testat 2 \\
		\hline
		\multicolumn{2}{|p{0.95\textwidth}|}{
			Überblick: \newline
			$\boxtimes$ Ziele erreicht \newline 
			$\square$ Es gibt einige Fragezeichen \newline
			$\square$ Fortschritt blockiert
		} \\
		\hline
		\multicolumn{2}{|p{0.95\textwidth}|}{
			Inhalt: \newline
			\textbf{Korrekturen Testat 2}
			\begin{itemize}
				\item Wie Motoren angesteuert werden auch ins Konzept
				\item Konzeptwahl gleich nach Einleitung
			\end{itemize}
			\textbf{Ausblick Testat 3}
			\begin{itemize}
				\item Zwei bis drei Algorithmen zu Bilderkennung unter verschiedenen Umgebungsbedingungen testen
				\item Bei Bildalgorithmen auf Pi: Hat keine Floating Point Unit kann zu Problemen führen
				\item Motoren-Typ auswählen und Ansteuerung definieren
				\item Geschwindigkeit von Rad messen mit Encoder
				\item CAD-Zeichnungen erstellen
			\end{itemize}
			Gruppe 39 erhält das Testat wenn die oben genannten Punkte korrigiert worden sind.
		} \\
		\hline
	\end{tabularx}
\end{table}
