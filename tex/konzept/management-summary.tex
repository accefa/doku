\section{Management Summary}
Die vorliegende Dokumentation befasst sich mit dem Lösungskonzept der Gruppe 39 für eine autonome Ballwurfmaschine, die im Modul PREN hergestellt wird. 
Im Rahmen des Moduls wurde ein Konzept erarbeitet, welches die Aufgabenstellung bestmöglich erfüllt. Dazu wurden diverse Prototypen und Teilfunktionsanalysen erstellt und ausgewertet. In der Dokumentation wird zuerst auf die Hauptfunktionen und danach auf die einzelnen Hauptkomponenten eingegangen. Detaillierte Versuchsauswertungen und Berechnungen sind dem Anhang angefügt.  
Die Arbeit hat ergeben, dass die Aufgabenstellung mit einem stationären System erfüllt werden kann. Die Ballwurfmaschine wird nur um eine Achse gedreht, um den Abwurfmechanismus in Richtung des Korbes auszurichten. Die Ortung des Korbes geschieht mittels Bildbearbeitung und die Bälle werden mit einem Drehrad beschleunigt. 
Das weitere Vorgehen besteht nun darin, den erarbeiteten Lösungsvorschlag zu verwirklichen. Dazu müssen die Komponenten möglichst kostengünstig eingekauft, die Software erstellt und die Schnittstellen optimal ausgelegt werden. Probleme können beim Zusammenspiel der Teilfunktionen entstehen. Dies muss mit einer offenen Kommunikation und einer sorgfältigen und überlegten Vorgehensweise verhindert werden. 
