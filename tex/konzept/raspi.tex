Der Raspberry Pi ist das zentrale Element des Systems. Der Raspberry Pi ist nichts anderes als ein Mini-Computer. Ergänzt wird der Raspberry Pi mit einem WLAN-Adapter, damit die drahtlose Kommunikation gewährleistet werden kann.

Der Mini-Computer ist wie folgt ausgestattet (Modell B+):
\begin{itemize}
	\item CPU: 700Mhz Broadcom BCM2835
	\item RAM: 512 MB
	\item 4 USB Ports
	\item 40 GPIO Pins
	\item Micro SD Slot (mit 8GB Micro SD)
	\item CSI Schnittstelle
\end{itemize}

Der Raspberry Pi kann grundsätzlich zu viel. Die CPU und RAM sind für heutige Verhältnisse nicht gerade berauschend, jedoch genügen sie den Anforderungen. Im Notfall können ressourcenintensive Berechnungen auf die externe Steuerungseinheit ausgelagert werden. Ein USB Port wird durch den WLAN-Adapter besetzt. Ein weiterer USB-Port wird durch das Freedomboard belegt. Die CSI-Schnittstelle dient dazu die PI Camera anzuhängen \cite{raspberri-b-plus-spec}.

Auf der SD Karte wird das Betriebssystem Arch Linux installiert. Diese Linux-Distribution ist minimalistisch ausgestattet. Es braucht lediglich 64 MB RAM und weniger als 800 MB Speicherplatz. So können die Hardware-Ressourcen, welche begrenzt sind, für das wirklich wichtige eingesetzt werden \cite{arch-linux-system-requirements}.
