\newpage
\subsubsection{Drehräder-Modell}
\begin{figure}[h!]
	\centering
	\begin{subfigure}{.4\textwidth}
		\centering
		\includegraphics[width=0.6\textwidth]{../../fig/Versuch_Drehrad.png}
		\caption{Schrägen Wurf von der Seite}
		\label{fig:Aufbau der Versuch}
	\end{subfigure} %
	\begin{subfigure}{.4\textwidth}
		\centering
		\includegraphics[width=0.6\textwidth]{../../fig/Drehrad_1.jpg}
		\caption{Schrägen Wurf von Oben}
		\label{fig:Drehrad}
	\end{subfigure}
	\caption{Berechnung der Schräger Wurf}
	\label{Drehrad Versuch}
\end{figure}
Mit diesem Versuch sollte festgestellt werden, von welcher Seite (Ober/ Unterseite) her, der Ball beschleunigt werden soll. Zudem sollte der Einfluss vom Drall ermittelt werden. Wir haben einen Polypropylen Rad mit einem Durchmesser von 15cm. verwendet. Als Antrieb wurde eine Akkubohrmaschine eingesetzt. Bei diesem Versuchsaufbau spielte die erreichte Wurfdistanz eine untergeordnete Rolle. Viel wichtiger war die Treffgenaugkeit.  
Der Versuch wurde auf zwei verschiedene Arten durchgeführt: ein erstes mal mit der Drehrad oberhalb des Balles, und ein zweites mit der Drehrad unten. Mit dieser unterschiedlichen Versuchsaufbauarten konnte die Auswirkungen auf die Treffgenaugkeit und die Flugbahn bestimmt werden.\\ \\
Nach mehrere Versuche wurde festgelegt, dass beide Varianten recht zuverlässig funktionieren. Da der Prototyp relativ instabil war, haben wir mit schlechteren Ergebnissen gerechnet. Die Bälle erreichten eine Distanz von ca. 1.3 Meter und landeten innerhalb von einem von einem Kreis mit einem Durchmesser alle Bälle sind innerhalb einem Kreis mit einem Durchmesser????????????????????????????????? kleiner als 30 cm gelandet. Mit einige Anpassungen die an die Maschine gemacht werden können, wie zum Beispiel eine genauere Vorrichtung und ein Rad aus weicheres Material, kann die Genauigkeit noch erhöht werden.\\ Als weiteres haben wir bemerkt das der Wurf mit der Drehrad Unter der Ball etwas präziser ist. Weiterhin haben wir notiert das der Drehung der Ball keine Auswirkung auf der Flugbahn hat. Das wurde erwartet, da die Geschwindigkeit der Wurf und der Drehung klein ist. \\

