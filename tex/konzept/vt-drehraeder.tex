\newpage
\subsubsection{Drehräder-Modell}
Mit diesem Versuch sollte festgestellt werden, von welcher Seite (Ober- oder Unterseite) her, der Ball beschleunigt werden soll. Zudem soll der Einfluss vom Drall ermittelt werden. Wir haben einen Polypropylen Rad mit einem Durchmesser von 15cm verwendet. Als Antrieb wurde eine Akkubohrmaschine eingesetzt. Bei diesem Versuchsaufbau spielte die erreichte Wurfdistanz eine untergeordnete Rolle. Viel wichtiger war die Treffgenauigkeit.  
Der Versuch wurde auf zwei verschiedene Arten durchgeführt: zuerst mit dem Drehrad oberhalb des Balles, und ein zweites Mal mit dem Drehrad auf der Unterseite. Mit diesen unterschiedlichen Versuchsaufbauarten konnte die Auswirkungen auf die Treffgenauigkeit und die Flugbahn bestimmt werden.

Nach mehrere Versuche wurde festgelegt, dass beide Varianten zuverlässig funktionieren. Da der Prototyp relativ instabil war, haben wir mit schlechteren Ergebnissen gerechnet. Die Bälle erreichten eine Distanz von ca. 1.3 Meter und landeten innerhalb von einem von einem Kreis mit einem Durchmesser von 30cm. 

Zudem wurde festgestellt, dass der Wurf mit dem Drehrad auf der Unterseite etwas präziser ist. Der Grund liegt darin, dass der Ball über das Drehrrad hinaus geführt wird. Befindet sich die Führung oberhalb des Balles, kann sich dieser durch die Schwerkraft selber von der Führung lösen. Befindet sich diese Führung auf der Unterseite, wird der Ball in seiner Flugbahn nach dem Abschuss behindert. Der Versuchsaufbau hat ergeben, dass der Drall keinen signifikanten Einfluss auf die Flugbahn hat.

\begin{figure}[h!]
	\centering
	\begin{subfigure}{.4\textwidth}
		\centering
		\includegraphics[width=0.6\textwidth]{../../fig/Versuch_Drehrad.png}
		\caption{Prototyp}
		\label{fig:Aufbau der Versuch}
	\end{subfigure} %
	\begin{subfigure}{.4\textwidth}
		\centering
		\includegraphics[width=0.6\textwidth]{../../fig/Drehrad_1.jpg}
		\caption{Drehrad}
		\label{fig:Drehrad}
	\end{subfigure}
	\caption{Versuchsaufbau mit einem Drehrad}
	\label{Drehrad Versuch}
\end{figure}
