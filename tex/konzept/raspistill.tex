\subsubsection{raspistill}
\label{subsub:raspistill}
Die Kamera wird über das mitgelieferte Kommandozeilenprogramm \verb|raspistill| angesteuert. Das Programm bietet zahlreiche Einstellungsmöglichkeiten. Eine kleine, für das Projekt relevante Auswahl ist in der Tabelle \ref{tab:raspistill} aufgeführt. Eine ausführliche Dokumentation des Programms kann man unter folgendem Link abrufen: \\

\href{http://www.raspberrypi.org/wp-content/uploads/2013/07/RaspiCam-Documentation.pdf}{http://www.raspberrypi.org/wp-content/uploads/2013/07/RaspiCam-Documentation.pdf} \\

\begin{table}[h!]
	\renewcommand{\arraystretch}{1.5}
	\begin{tabular}{|l|p{14cm}|}
		\hline Option & Beschreibung \\ 
		\hline \verb|--nopreview| & Deaktiviert die Bildvorschau (Optimierung) \\ 
		\hline  \verb|--quality 50| & Tests haben gezeigt, dass die Qualität von 50 genügt. Die Bildgrösse wird somit einiges kleiner. \\ 
		\hline  \verb|--roi| & Bildausschnitt festlegen \\ 
		\hline  \verb|--contrast| & Kontrast einstellen \\ 
		\hline  \verb|-cfk 128:128| & Erzeugt ein Bild in Graustufen \\ 
		\hline 
	\end{tabular} 
	\caption{Relevante Einstellungen von raspistill}
	\label{tab:raspistill}
\end{table}

Ein Aufruf von \verb|raspistill| könnte wie folgt aussehen (wird in diesem Projekt nicht verwendet): \\

\verb|raspistill --nopreview --quality 50 -roi 0.5,0.5,0.25,0.25 --contrast 50 -cfx 128:128| \\

Das Programm speichert das Bild anschliessend in einer jpg-Datei ab. Für die weitere Verarbeitung des Bildes ist das nicht optimal. Das Bild müsste vom Python-Programm, welches für die Erkennung des Korbes zuständig ist, nochmals eingelesen werden. Deshalb wird \verb|raspistill| direkt aus dem Erkennungs-Programm über ein Python-Interface angesteuert. Dieses Interface wird von der Bibliothek \verb|picamera|\footnote{\href{http://picamera.readthedocs.org/en/release-1.8/index.html}{http://picamera.readthedocs.org/en/release-1.8/index.html}} zur Verfügung gestellt und ermöglicht ein Objekt der Kamera zu erstellen. Das Kamera-Objekt liefert direkt ein Bild als PIL\footnote{Python Image Library}-Objekt zurück. Das PIL-Objekt kann anschliessend für die weitere Verarbeitung in Python genutzt werden. Auflistung \ref{lst:picamera} zeigt ein Codeausschnitt wie die Kamera in diesem Projekt angesteuert wird.

\begin{lstlisting}[language=Python,caption={Bild aufnehmen mit der Raspberry Pi Camera},label=lst:picamera]
import io
import time
import picamera
from PIL import Image

# Create the in-memory stream
stream = io.BytesIO()
with picamera.PiCamera() as camera:
camera.start_preview()
time.sleep(2)
camera.capture(stream, format='jpeg')
# "Rewind" the stream to the beginning so we can read its content
stream.seek(0)
image = Image.open(stream)
\end{lstlisting}
