\subsubsection{LowLevel}
Als LowLevel Schnittstellen werden die einfachen Ansteuerungen
bezeichnet, welche direkt mittels eines Miktrocontroller erzeugt
werden.

Die einzelenen Hardwarekomponenten wie Motorentreiber werden mittels
einfachen digitalen Signalen bedient. Dies können einfache digitale
Pegel sein, standardisierte Busse wie I$^2$C oder spezielle Signale
wie PWM. 

Da noch nicht alle Hardwarekomponenten definiert sind, gibt es keine
definitive Aussage über die verwendeten Signale bzw. Busse. Vorzusehen
sind sicherlich I$^2$C und SPI, da diese weit verbreitete Schnittstellen
sind, durch das Freedomboard unterstützt werden und deren
Implementierung relativ einfach ist.
