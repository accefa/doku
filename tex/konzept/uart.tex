Die UART Schnittstelle dient der Kommunikation zwischen Raspberry Pi und
dem Freedomboard, welches die LowLevel Funktionalität der Maschine 
umsetzt. Dieses implementiert sämtliche Ansteuerng von Hardwarekomponenten
wie Motorentreibern oder Sensoren. Die übergeordnete Instanz des Freedomboards
ist das Raspberry Pi, welches die Schnittstelle zum Benutzer bildet. Die für
diese Abstraktion notwendige Schnittstelle wird mittels UART
implementiert. UART\footnote{Universal Asynchronous Receiver Transmitter} ist ein weit verbreiteter Standard für serielle Schnittstellen.  Ein mögliches Protokoll ist in der Tabelle \ref{tab:uart}
zusammengefasst.

Die serielle Schnittstelle wird vom Freedomboard direkt implementiert auf
USB mittels einer USB-Seriell Wandlung. Dies ermöglicht es, das Freedomboard
direkt per USB-Kabel am Raspberry Pi zu verbinden.

\begin{table}[h!]
	\centering
	\begin{tabular}{l l l l l}
		Aktion & Message & Parameter & Antwort \\
		\hline
		drehen um $\varphi$ 
			& \verb!rotate!
			& Winkel $\varphi$
			& aktuelle Position \\
		Entfernung $s$ setzen
			& \verb!distance!
			& Entfernung $s$
			& aktuelle Entfernung \\
		Ausrichtungsmotor enable
			& \verb!rotater!
			& enable/disable
			& aktueller status\\
		Schussmotor enable
			& \verb!shooter! 
			& enable/disable
			& aktueller Status \\
		Lademotor enable
			& \verb!loader!
			& enable/disable
			& aktueller status\\
		Schussabgabe
			& \verb!fire!
			& -
			& ok/nok\\
		Reset
			& \verb!reset!
			& -
			& -\\
	\end{tabular}
	\caption{Protokollentwurf der seriellen Schnittstelle}
	\label{tab:uart}
\end{table}
