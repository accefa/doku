\subsection{Softwaresysteme}
Mit Software wird die Kommunikation, die Ortung des Korbes sowie die Steuerung umgesetzt. Dazu werden in den folgenden Kapitel die Architektur der Software sowie einzelne Algorithmen beschrieben.

\subsubsection{Kommunikation}
Die Kommunikation erfolgt über Wireless LAN (WLAN). Das Raspberry PI auf der Ballwurfmaschine stellt einen Access Point zur Verfügung mit welchem sich externe Steuerungseinheiten verbinden können. Die primäre externe Steuerungseinheit ist ein Notebook.\\
Auf dem PI werden Webservices angeboten, welche von dem Notebook aufgerufen werden können. Jede gängige Programmiersprache bietet Unterstützung für Webservices an. Dadurch schränkt man die Wahl der externen Steuerungseinheit nicht ein. Diese muss einzig TCP/IP-Stack unterstützen.\\
\\
Das Notebook registriert sich zu Beginn beim Raspberry PI für verschiedene Operationen. Das Notebook soll folgende Meldungen vom PI während des Prozesses erhalten: Aktuelle Tätigkeiten (Protokoll), Berechnungsaufgaben und zum Schluss das Stoppsignal. Das Stoppsignal wird explizit ausgelagert, da es eine grosse Bedeutung im Ablauf hat. Sobald dieses Signal an das Notebook gelangt, wird die Zeit gestoppt. Für all diese Informationen muss sich das Notebook vorgängig beim PI registrieren. Diese Granularität ermöglicht es, dass beispielsweise ein Mainframe explizit für die Berechnung eingesetzt wird oder dass auf einem Mobile-Device nur das Protokoll angezeigt wird.
\\
Auf dem Raspberry PI erfolgt die Implementierung in Python. Mittels Python lassen sich Webservices realisieren und gleichzeitig bietet es eine sehr einfache API um auf serielle Schnittstellen zuzugreifen. Allfällige Inputs vom Web-Interface können so direkt an die Kamera und Motoren weitergeleitet werden. Auf dem PI werden zwei Schnittstellen angeboten. Zum einen das Registrations-Interface und zum anderen die Operate-Schnittstelle.\\
\\
Die externe Steuerungseinheit ist völlig unabhängig vom PI und muss lediglich das Client-Interface als Webinterface implementieren sofern es sich für die gegeben Funktionen am PI registriert.\\
\\
Architektur\\
TODO BILD
\\
Schnittstellenbeschreibung

Schnittstelle Registration
- registerStopp
- registerCalculation
- registerProtocol

Schnittstelle Operate
- sendStartSignal
- sendCalculationResults

Schnittstelle Client
- receiveStopp
- receiveCalculation
- receiveProtocoll

\subsubsection{Ortung des Korbes}
Der Ort des Korbes wird optisch bestimmt. Dazu wird zu Beginn des Prozesses ein Foto geschossen. Dieses Bild wird mittels eines Algorithmus analysiert um die Koordinaten des Korbes zu bestimmen.\\
\\
Algorithmus\\
Es wurden zwei Ideen für die Entwicklung des Algorithmus verfolgt. Der Eine basiert auf eigenen Überlegungen und der Andere auf bestehende Kanten- bzw. Objekterkennung. Die grösste Herausforderung stellen die unterschiedlichen Lichtverhältnisse dar. Der Algorithmus muss so stabil sein, dass dieser damit umgehen kann. Der Aufbau wird mit Licht von oben beschienen. Auch mit Licht von der Seite durch die Fenster ist zu rechnen.\\
\\
Algorithmus: Eigene Überlegungen\\
Im ersten Schritt wird das Bild zugeschnitten. Dies hat den Vorteil, dass irrelevante Informationen verschwinden so wird das Bild physisch kleiner und auch die nächsten Verarbeitungsschritte sind schneller. Im zweiten Schritt wird der Kontrast des Bildes verstärkt. Das Histogramm verbreitert sich und hellere Punkte werden Heller und dunklere Punkte noch dunkler. Dann folgt die Konvertierung des Bildes nach Graustufen. Jedes Pixel hat einen gleich grossen Anteil an R-, G- und B-Werten. Zum Schluss wird eine einzelne Pixel-Linie analysiert. Dabei wird ersichtlich, dass es einen Bereich gibt, wo mehrere Pixel mit einem sehr tiefen RGB Wert aufeinander folgen. Umso tiefer die RGB Werte sind umso dunkler ist es dort (RGB(0,0,0) = black).
\\
Algorithmus: Kanten- und Objekterkennung\\
TODO

\subsubsection{Steuerung}