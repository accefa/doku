\section{Projektmanagement Planung}
Für die Organisation des Projektes wurde der webbasierte Hosting-Dienst Github verwendet. Auf einem über Github erstellten Verzeichnis (Repository genannt) werden alle Projektrelevanten Daten zentral gespeichert und sind so für alle Teammitglieder über das Internet ständig zugänglich. Die Verwaltung der Daten erfolgt über ein gleichnamiges Programm. Werden Änderungen an einer Datei vorgenommen, erkennt diese das Programm jeweils, womit sich Konflikte (zB ungewolltes Überschreiben von Daten) grösstenteils verhindern lassen.

Github bietet dem Nutzer das Erstellen von "Issues" an, die jeweils an Teammitglieder zugewiesen werden können. Diese "Issues" ermöglichen es, Aufgaben einfach und effektiv zu verteilen.

Die Aufgabenstellung dieses Projektes beinhaltete die Abgabe von 3 Berichten über den Verlauf des Semesters hinweg. Mittels Github wurde für jede Abgabe jeweils ein "Milestone" erstellt, welcher die zu erledigenden Aufgaben bzw. "Issues" für die Berichte zusammenfasst. So weiss jeder Nutzer jeweils Bescheid was er bis zu Abgabe zu tun hat.
\subsection{Projekteam}

\begin{tabularx}{\columnwidth}{XX}
	
	\textbf{Adriano Valsangiacomo} \newline
	Maschinentechnik \newline
	\href{mailto:adriano.valsangiacomo@stud.hslu.ch}{adriano.valsangiacomo@stud.hslu.ch} \newline
	
	&  
	
	\textbf{Ervin Mazlagi\'c} \newline
	Elektrotechnik \newline
	\href{mailto:ervin.mazlagic@stud.hslu.ch}{ervin.mazlagic@stud.hslu.ch} \newline 
	
	\\ 
	
	\textbf{Christian Spycher} \newline
    Maschinentechnik \newline
    \href{mailto:christian.spycher@stud.hslu.ch}{christian.spycher@stud.hslu.ch} \newline 
     
    
    & 
    
    \textbf{Fabian Wüthrich} \newline
    Informatik \newline
    \href{mailto:fabian.wuethrich.01@stud.hslu.ch}{fabian.wuethrich.01@stud.hslu.ch} \newline 
    
    \\ 
	
	\textbf{Christian Schürch} \newline
	Maschinentechnik \newline
	\href{mailto:christian.schuerch@stud.hslu.ch}{christian.schuerch@stud.hslu.ch} \newline 
	
	& 
	 
	\textbf{Alexander Suter} \newline
	Informatik \newline
	\href{mailto:alexander.suter@stud.hslu.ch}{alexander.suter@stud.hslu.ch} \newline 
	
	\\ 
\end{tabularx} 


\subsection{Organigramm}
Die Hierarchie des Projektteams ist im Organigramm in Abbildung \ref{fig:organigramm} dargestellt.

\begin{figure}[h!]
	\centering
	\begin{tikzpicture}
	\node [draw](projektleiter) {
		\begin{tabular}{c}
		Projektleiter \\
		Adriano \\
		Valsangiacomo
		\end{tabular} 
	};
	
	\node [draw, below left =1cm and 2cm of  projektleiter] (informatik) {Informatik} edge [<-] (projektleiter);
	\node [draw, below =1cm of  projektleiter] (elektrotechnik) {Elektrotechnik} edge [<-] (projektleiter);
	\node [draw, below right =1cm and 2cm of  projektleiter] (mechanik) {Mechanik} edge [<-] (projektleiter);
	
	\node [draw, below left  = of  informatik] (fabianwuethrich) {
			\begin{tabular}{c}
			Fabian \\
			Wüthrich
			\end{tabular}
		} edge [<-] (informatik);
	\node [draw, below right = of  informatik] (alexsuter) {
			\begin{tabular}{c}
			Alexander \\
			Suter
			\end{tabular} 
		} edge [<-] (informatik);
	
	\node [draw, below =3cm of  elektrotechnik] (ervinmazlagic) {
			\begin{tabular}{c}
			Ervin \\
			Mazlagi\'c
			\end{tabular}
		} edge [<-] (elektrotechnik);
	
	\node [draw, below left = of  mechanik] (christianspycher) {
		\begin{tabular}{c}
		Christian \\
		Spycher
		\end{tabular} 
		} edge [<-] (mechanik);
	\node [draw, below = of  mechanik] (christianschuerch) {
		\begin{tabular}{c}
		Christian \\
		Schürch
		\end{tabular}} edge [<-] (mechanik);
	\node [draw, below right = of  mechanik] (adrianovalsangiacomo) {
				\begin{tabular}{c}
				Adriano \\
				Valsangiacomo
				\end{tabular}  
		  } edge [<-] (mechanik);
	\end{tikzpicture}
	\caption{Organigramm}
	\label{fig:organigramm}
\end{figure}


\begin{landscape}
\section{Risikoanalyse}
\begin{table}[h!]
    \centering
    \begin{tabular}{p{0.1\textwidth} c p{0.15\textwidth} p{0.2\textwidth} p{0.15\textwidth} p{0.1\textwidth} p{0.2\textwidth}}
		& Nr. & Risiko & Ursache & Wahrscheinlichkeit & Auswirkung & Massnahmen \\
        \hline \hline
        & & & & & & \\
        \rowcolor{yellow} 
        Personelle Faktoren & 1 & Falscher Projektleiter & Falsche Vorstellungen vom Projekt & Vorstellbar & gering & Erfolg darf nicht vom Projektleiter abhängen \\ 
        \rowcolor{yellow}
	    & 2 & Zwei Mitglieder einer Studienrichtung fallen aus & Krankheit, Studienabbruch & Unwahrscheinlich & Katastrophal & externe Hilfe holen \\
        \rowcolor{green}
        & 3 & Betreuender Dozent fällt aus & Krankheit, beschäftigt mit anderen Projekten & vorstellbar & unwesentlich &
	\end{tabular}
\end{table}
\end{landscape}


\subsection{Risiko-Bewertungsschema}
\begin{table}[h!]
	\renewcommand{\arraystretch}{1.5}
	\centering
	\begin{tabular}{r || c c c c}
		häufig 		
			& \cellcolor{red} 
			& \cellcolor{red}
			& \cellcolor{red}
			& \cellcolor{red} \\
		wahrscheinlich		
			& \cellcolor{yellow} 
			& \cellcolor{yellow} 
			& \cellcolor{red}
			& \cellcolor{red} \\
		gelegentlich		
			& \cellcolor{yellow}
			& \cellcolor{yellow}
			& \cellcolor{yellow}
			& \cellcolor{red} \\
		vorstellbar		
			& \cellcolor{green}
			& \cellcolor{yellow}
			& \cellcolor{yellow}
			& \cellcolor{yellow} \\
		unwahrscheinlich	
			& \cellcolor{green}
			& \cellcolor{green}
			& \cellcolor{yellow}
			& \cellcolor{yellow} \\
		unvorstellbar		
			& \cellcolor{green}
			& \cellcolor{green}
			& \cellcolor{green}
			& \cellcolor{green} \\
		\hline
		& unwesentlich & geringfügig & kritisch & katastrophal
	\end{tabular}
\end{table}
