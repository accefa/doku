\section{Projektmanagement Planung}
Für die Organisation des Projektes wurde der webbasierte Hosting-Dienst Github verwendet. Auf einem über Github erstellten Verzeichnis (Repository genannt) werden alle Projektrelevanten Daten zentral gespeichert und sind so für alle Teammitglieder über das Internet ständig zugänglich. Die Verwaltung der Daten erfolgt über ein gleichnamiges Programm. Werden Änderungen an einer Datei vorgenommen, erkennt diese das Programm jeweils, womit sich Konflikte (zB ungewolltes Überschreiben von Daten) grösstenteils verhindern lassen.

Github bietet dem Nutzer das Erstellen von \emph{Issues} an, die jeweils an Teammitglieder zugewiesen werden können. Diese \emph{Issues} ermöglichen es, Aufgaben einfach und effektiv zu verteilen.

Die Aufgabenstellung dieses Projektes beinhaltete die Abgabe von 3 Berichten über den Verlauf des Semesters hinweg. Mittels Github wurde für jede Abgabe jeweils ein \emph{Milestone} erstellt, welcher die zu erledigenden Aufgaben bzw. \emph{Issues} für die Berichte zusammenfasst. So weiss jeder Nutzer jeweils Bescheid was er bis zu Abgabe zu tun hat. \\

\subsection{Kostenüberischt}
Für unseres Projekt hatten wir ein Budget von 600 CHF. Um unsere Kosten im Augenblick zu halten, haben wir sie in einer Tabelle eingefügt. In die Kolonne Budget haben wir eingetragen, was wir schätzen das dieser Teil kosten könnte, und damit konnten wir auch ein Überblick auf die gesamte Verteilung der Kosten erhalten. Die eingetragten Werten sollten, nach eine Recherche in Internet, als maximales Preis behandelt werden.\\
\begin{figure}[h!]
	\center
	\includegraphics[width=0.8\textwidth]{../../fig/Kostenübersicht.jpg}
	\caption{Kostenübersicht}
	\label{fig:Kostenübersicht}
\end{figure}\\
Von bei unseres 600 CHF Budget, haben wir in der Tabelle 500 davon benutzt. Der Restlichem 100 planen wir das es für andere Teilen (wie schrauben, Kabeln, usw.) benutzt werden. \\
Ins gesamt haben wir während der Modul PREN 1, haben wir von unser zur verfügung 200, 85.40 CHF benutzt. 

\subsection{Projekteam}

\begin{tabularx}{\columnwidth}{XX}
	
	\textbf{Adriano Valsangiacomo} \newline
	Maschinentechnik \newline
	\href{mailto:adriano.valsangiacomo@stud.hslu.ch}{adriano.valsangiacomo@stud.hslu.ch} \newline
	
	&  
	
	\textbf{Ervin Mazlagi\'c} \newline
	Elektrotechnik \newline
	\href{mailto:ervin.mazlagic@stud.hslu.ch}{ervin.mazlagic@stud.hslu.ch} \newline 
	
	\\ 
	
	\textbf{Christian Spycher} \newline
    Maschinentechnik \newline
    \href{mailto:christian.spycher@stud.hslu.ch}{christian.spycher@stud.hslu.ch} \newline 
     
    
    & 
    
    \textbf{Fabian Wüthrich} \newline
    Informatik \newline
    \href{mailto:fabian.wuethrich.01@stud.hslu.ch}{fabian.wuethrich.01@stud.hslu.ch} \newline 
    
    \\ 
	
	\textbf{Christian Schürch} \newline
	Maschinentechnik \newline
	\href{mailto:christian.schuerch@stud.hslu.ch}{christian.schuerch@stud.hslu.ch} \newline 
	
	& 
	 
	\textbf{Alexander Suter} \newline
	Informatik \newline
	\href{mailto:alexander.suter@stud.hslu.ch}{alexander.suter@stud.hslu.ch} \newline 
	
	\\ 
\end{tabularx} 


\subsection{Organigramm}
Die Hierarchie des Projektteams ist im Organigramm in Abbildung \ref{fig:organigramm} dargestellt.

\begin{figure}[h!]
	\centering
	\begin{tikzpicture}
	\node [draw](projektleiter) {
		\begin{tabular}{c}
		Projektleiter \\
		Adriano \\
		Valsangiacomo
		\end{tabular} 
	};
	
	\node [draw, below left =1cm and 2cm of  projektleiter] (informatik) {Informatik} edge [<-] (projektleiter);
	\node [draw, below =1cm of  projektleiter] (elektrotechnik) {Elektrotechnik} edge [<-] (projektleiter);
	\node [draw, below right =1cm and 2cm of  projektleiter] (mechanik) {Mechanik} edge [<-] (projektleiter);
	
	\node [draw, below left  = of  informatik] (fabianwuethrich) {
			\begin{tabular}{c}
			Fabian \\
			Wüthrich
			\end{tabular}
		} edge [<-] (informatik);
	\node [draw, below right = of  informatik] (alexsuter) {
			\begin{tabular}{c}
			Alexander \\
			Suter
			\end{tabular} 
		} edge [<-] (informatik);
	
	\node [draw, below =3cm of  elektrotechnik] (ervinmazlagic) {
			\begin{tabular}{c}
			Ervin \\
			Mazlagi\'c
			\end{tabular}
		} edge [<-] (elektrotechnik);
	
	\node [draw, below left = of  mechanik] (christianspycher) {
		\begin{tabular}{c}
		Christian \\
		Spycher
		\end{tabular} 
		} edge [<-] (mechanik);
	\node [draw, below = of  mechanik] (christianschuerch) {
		\begin{tabular}{c}
		Christian \\
		Schürch
		\end{tabular}} edge [<-] (mechanik);
	\node [draw, below right = of  mechanik] (adrianovalsangiacomo) {
				\begin{tabular}{c}
				Adriano \\
				Valsangiacomo
				\end{tabular}  
		  } edge [<-] (mechanik);
	\end{tikzpicture}
	\caption{Organigramm}
	\label{fig:organigramm}
\end{figure}


\begin{landscape}
\subsection{Risikoanalyse}
\begin{table}[h!]
    \small
    \centering
    \begin{tabular}{p{0.08\textwidth} c p{0.15\textwidth} p{0.4\textwidth} p{0.15\textwidth} p{0.1\textwidth} p{0.2\textwidth}}
		& Nr. & Risiko & Ursache & Wahrscheinlichkeit & Auswirkung & Massnahmen \\
        \hline \hline
        & & & & & & \\
        \rowcolor{yellow} Personelle Faktoren 
            & 1 
            & Falscher Projektleiter 
            & Falsche Vorstellungen vom Projekt 
            & Vorstellbar 
            & gering 
            & Erfolg darf nicht vom Projektleiter abhängen \\ 
        \rowcolor{yellow}
	        & 2 
            & Zwei Mitglieder einer Studienrichtung fallen aus 
            & Krankheit, Studienabbruch 
            & Unwahrscheinlich 
            & Katastrophal 
            & externe Hilfe holen \\
        \rowcolor{green}
            & 3 
            & Betreuender Dozent fällt aus 
            & Krankheit, beschäftigt mit anderen Projekten 
            & vorstellbar 
            & unwesentlich 
            & \\
        \rowcolor{yellow}
            & 4	
            & Team verstreitet sich	
            & unterschiedliche Ansichten, falsche Arbeitsplanung, unterschiedlicher Arbeitseinsatz, fehlende Verbindlichkit	
            & Vorstellbar 
            & Katastrophal 
            & Arbeit genau Planen, offene Kommunikation, Verbindlichkeiten schafen \\
        \rowcolor{yellow}
            & 5	
            & Ungenügende Kommunikation 
            & ungenügende Planung, fehlende Interdisziplinarität, schlechte Stimmung im Team 
            & Vorstellbar 
            & Katastrophal 
            & offen und oft Kommunizieren, Rückmeldungen geben \\
        \rowcolor{yellow}
            & 6	
            & 
            & 
            & 
            & 
            & \\
        \rowcolor{yellow} Zeitliche Faktoren 
            & 7 
            & Zu optimistische Ressourcenplanung 
            & hohe Belastung durch das Studium, zu komplexe Lösungsvarianten 
            & Vorstellbar 
            & Kritisch 
            & realistisch Planen, komplexität verringern \\
	    \rowcolor{yellow}
            & 8 
            & mangelnde Produktivität 
            & Überforderung, schlechte Planung, desinteresse, fehlende Zeitplanung 
            & Unwahrscheinlich 
            & Kritisch 
            & gute Planung, Verbindlichkeiten schafen \\
        \rowcolor{yellow}
	        & 9 
            & Ungenügende Zeitkontrolle 
            & unklare Zeitplanung 
            & Vorstellbar 
            & Kritisch 
            & Reflektion der Arbeiten, Meilensteine festlegen \\
        \rowcolor{yellow}
	        & 10 
            & Unklare Zeitplanung 
            & Zeitplanung unverbindlich oder nicht vorhanden 
            & Vorstellbar 
            & Kritisch 
            & Realistisch Planen, verbindliche Zeitplanung \\
        \rowcolor{yellow}
	        & 11 
            & 
            & 
            & 
            & 
            & \\
        \rowcolor{red} Technische Faktoren 
            & 12 
            & Unerwarteter Datenverlust 
            & Diebstall, Unachtsamkeit, schlechte Organisation 
            & Gelegentlich 
            & Katastrophal 
            & Arbeit auf mehreren Platformen speichern \\
        \rowcolor{green} 
            & 13 
            & Fehlende Fachkentnisse 
            & zu hohe Ansprüche 
            & Unwahrscheinlich 
            & Geringfügig 
            & externe Hilfe hohlen, komplexität verringern \\
        \rowcolor{yellow}
	        & 14 
            & Nicht zu resultierender Lösungsvorschlag 
            & zu hohe Ansprüche, fehlende Prototypen, schlechte Planung 
            & Unwahrscheinlich 
            & Kritisch 
            & Prototypen erstellen, komplexität verringern \\
        \rowcolor{yellow} 
            & 15 
            & 
            & 
            & 
            & 
            & \\
	\end{tabular}
\end{table}
\end{landscape}


\subsubsection{Risiko-Bewertungsschema}
\begin{table}[h!]
	\renewcommand{\arraystretch}{1.5}
	\centering
	\begin{tabular}{r || c c c c}
		häufig 		
			& \cellcolor{red} 
			& \cellcolor{red}
			& \cellcolor{red}
			& \cellcolor{red} \\
		wahrscheinlich		
			& \cellcolor{yellow} 
			& \cellcolor{yellow} 
			& \cellcolor{red}
			& \cellcolor{red} \\
		gelegentlich		
			& \cellcolor{yellow}
			& \cellcolor{yellow}
			& \cellcolor{yellow}
			& \cellcolor{red} \\
		vorstellbar		
			& \cellcolor{green}
			& \cellcolor{yellow}
			& \cellcolor{yellow}
			& \cellcolor{yellow} \\
		unwahrscheinlich	
			& \cellcolor{green}
			& \cellcolor{green}
			& \cellcolor{yellow}
			& \cellcolor{yellow} \\
		unvorstellbar		
			& \cellcolor{green}
			& \cellcolor{green}
			& \cellcolor{green}
			& \cellcolor{green} \\
		\hline
		& unwesentlich & geringfügig & kritisch & katastrophal
	\end{tabular}
\end{table}
