\subsubsection{Start Signal}
Das Start Signal präsentiert sich als eigenständige Ressource und kann über eine eindeutige URI\footnote{Uniform Resource Identifier} angesteuert werden. Der Webserver auf dem Raspberry Pi stellt diese Ressource bereit. Soll überprüft werden ob bereits ein Startsignal gesendet wurde, kann ein GET-Request an den Webserver gesendet werden. Die Anfrage ist in der Tabelle \ref{tab:get-start-signal} beschrieben.

\begin{table}[h!]
	\centering
	\begin{tabular}{|l|l|}
		\hline Anfrage an Raspberry Pi 			 & Antwort von Raspberry Pi 	  \\ 
		\hline \verb|GET /start-signal HTTP/1.1| & \verb|200 OK| 				  \\
			   \verb|Host: 192.168.1.2| 		 & \verb|Content-Type: text/xml|  \\
			   									 &  							  \\
			   									 & \verb|<started>false</started>| \\ 
		\hline 
	\end{tabular} 
	\caption{GET-Request an Start Signal}
	\label{tab:get-start-signal}
\end{table}

Mittels eines PUT-Request an die Ressource kann der Ablauf gestartet werden. Ein Beispiel für so eine Request ist in der Tabelle \ref{tab:put-start-signal} beschrieben.

\begin{table}[h!]
	\centering
	\begin{tabular}{|l|l|}
		\hline Anfrage an Raspberry Pi 			 & Antwort von Raspberry Pi \\ 
		\hline \verb|PUT /start-signal HTTP/1.1| & \verb|200 OK| 			\\
			   \verb|Host: 192.168.1.2| 		 & 							\\
			   \verb|Content-Type: text/xml|	 &  						\\
			   									 &  						\\
			   \verb|<status>|	 				 &  						\\
			   \verb|<started>true</started>|	 &  						\\
			   \verb|<from>192.168.1.3</from>|	 &  						\\
			   \verb|</status>|	 				 &  						\\ 
		\hline 
	\end{tabular} 
	\caption{PUT-Request an Start Signal}
	\label{tab:put-start-signal}
\end{table}

Dem PUT-Request wird zusätzlich die Adresse der externen Steuereinheit mitgegeben. Diese Adresse wird benötigt um das Stop Signal zurückzusenden. Ist bei der Verarbeitung einer Anfrage ein Fehler aufgetreten, wird dieser in der Antwort des Webservers über HTML-Statuscodes beschrieben. Diese simple Schnittstelle deckt alle Anforderungen an das Start Signal ab.