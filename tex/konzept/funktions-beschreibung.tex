\subsection{Funktionsbeschreibung}
TODO Beschreibung der Funktionalität der einzelnen Blöcke und deren Beziehungen 
\subsubsection{Ballvorschub}
Die Bälle werden mit einem Bewegungsgewinde (Trapezgewinde) dem Drehrad zugeführt. Dabei ist die Trapezgewindespindel fest (je nach Untersetzung auch über ein Getriebe) mit dem Elektromotor verbunden. Die Linearbewegung wird von der Mutter ausgeführt. Um eine Beschädigung des Drehrades zu verhindern, wird die Linearbewegung mit einem Endschalter begrenzt. Sowohl in der unteren wie auch in der oberen Endposition. Der Elektromotor wird zwei Drehrichtungen besitzen, damit ein Nachladen ohne grossen Aufwand möglich ist. Die Bälle müssen sowohl gegen unten wie auch gegen oben geführt sein. Eine Führung auf der Oberseite wird nötig, weil die Bälle von der Unterseite hehr beschleunigt werden.
\subsubsection{Ausrichtung der Plattform}
Der Aufbau mit den Drehrädern und die Grundplatte werden voneinander getrennt sein, damit eine horizontale Ausrichtung möglich ist. Dabei wird der Schrittmotor fix mit der Platte mit dem Aufbau verbunden. An der Welle des Schrittmotors wird ein Ritzel befestigt. Auf der Grundplatte wird ein innenverzahntes Hohlrad befestigt. Das Hohlrad aus Kunststoff wird mit einem 3D Drucker hergestellt. Die Lagerung der Beiden Platten erfolgt im Zentrum mit einem Zapfen. An den Aussendurchmessern werden an drei oder vier Punkten Kunststoffelemente befestigt, damit die Platten seitlich abgestützt sind. Die Kunststoffelemente dienen als Gleitführung.
