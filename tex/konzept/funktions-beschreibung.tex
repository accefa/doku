\subsection{Funktionsbeschreibung}
Die Funktionen, welche im Ablaufdiagramm abgearbeitet werden (Abschnitt \ref{ss_ablaufdiagramm}), sind nachfolgend detaillierter beschrieben.

\subsubsection{Kommunikation zwischen externe Steuerungseinheit und Ballwurfmaschine}
Die Kommunikation erfolgt über Wireless LAN (WLAN). Das Raspberry PI, welches auf der Ballwurfmaschine installiert ist, stellt einen Access Point zur Verfügung mit welchem sich externe Steuerungseinheiten verbinden können. Die primäre externe Steuerungseinheit ist ein Notebook.\\
Auf dem PI werden Webservices angeboten, welche von dem Notebook aufgerufen werden können. Jede gängige Programmiersprache bietet Unterstützung für Webservices an. Dadurch schränkt man die Wahl der externen Steuerungseinheit nicht ein. Diese muss einzig den TCP/IP-Stack unterstützen. Die Webservices auf Seite des Raspberry PI werden mittels Python und mit Hilfe der Library web.py umgesetzt. Auf Seite der externen Steuerungseinheit gibt es eine Benutzeroberfläche, welche den Prozess anstossen kann (siehe in der UI-Skizze den Bereich Ballwurf in Abschnitt
\ref{ss-config-paramater-ortung-orb}).

\subsubsection{Ortung des Korbes}
Zur Ortung des Korbes wird eine Kamera und ein Erkennungsalgorithmus verwendet. Der Erkennungsalgorithmus wird vom Projektteam selbst entwickelt. Der Algorithmus ist in der Programmiersprache Python geschrieben und wird auf dem Raspberry Pi ausgeführt. Um den Korb zu orten, wird mit der Kamera ein Bild erstellt und an den Erkennungsalgorithmus weitergegeben. Der Algorithmus sucht den Korb und berechnet den Winkel welcher der Korb vom unteren Bildrand entfernt ist. Dieser Winkel übergibt er der UART-Schnittstelle (Abschnitt \ref{subsub:uart}), welche die Plattform ausrichtet und die Ball abwirft.

\subsubsection{Ausrichtung der Plattform}
Der Aufbau mit den Drehrädern und die Grundplatte werden voneinander getrennt sein, damit eine horizontale Ausrichtung möglich ist. Dabei wird der Schrittmotor fix mit der Platte mit dem Aufbau verbunden. An der Welle des Schrittmotors wird ein Ritzel befestigt. Auf der Grundplatte wird ein innenverzahntes Hohlrad befestigt. Das Hohlrad aus Kunststoff wird mit einem 3D Drucker hergestellt. Die Lagerung der Beiden Platten erfolgt im Zentrum mit einem Axiallager. Am Aussendurchmessern werden an drei oder vier Punkten Kunststoffelemente befestigt, damit die Platten seitlich abgestützt sind. Die Kunststoffelemente dienen als Gleitführung.

\subsubsection{Ballvorschub}
Die Bälle werden mit einem Bewegungsgewinde (Trapezgewinde) dem Drehrad zugeführt. Dabei ist die Trapezgewindespindel fest (je nach Untersetzung auch über ein Getriebe) mit dem Elektromotor verbunden. Die Linearbewegung wird von der Mutter ausgeführt. Um eine Beschädigung des Drehrades zu verhindern, wird die Linearbewegung mit einem Endschalter begrenzt. Sowohl in der unteren wie auch in der oberen Endposition. Der Elektromotor wird zwei Drehrichtungen besitzen, damit ein Nachladen ohne grossen Aufwand möglich ist. Die Bälle müssen sowohl gegen unten wie auch gegen oben geführt sein. Eine Führung auf der Oberseite wird nötig, weil die Bälle von der Unterseite hehr beschleunigt werden.

\subsubsection{Schussabgabe}
Die Tennisbälle werden mit einem Drehrad beschleunigt. Die Kraftübertragung auf die Tennisbälle erfolgt Kraftschlüssig. Das Drehrad überträgt die Kraft von der Unterseite hehr auf die Bälle. Auf der Oberseite werden die Bälle mit einem Federblech auf das Drehrad gedrückt.