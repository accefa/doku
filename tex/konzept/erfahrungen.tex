\subsection{Erfahrungen}
Alle Mitglieder des PREN Team 39 konnten während der Erarbeitung des Konzepts
wertvolle Erfahrung sammeln. Oft sind es die kleinen Dinge, welche einen
grossen Lerneffekt nach sich ziehen. Sei es lediglich das Zusammenarbeiten
in einem interdisziplinären Team, wobei man seinen eigenen Horizont erweitern
muss, um die Probleme der anderen Disziplinen zu verstehen.

\subsubsection*{Generell}
Das erste Mal mussten die meisten Mitglieder wirtschaftlich denken. Das
Budget ist nicht unbegrenzt und man kann jeweils nicht einfach die präziseste,
stärkste und schnellste Komponente einkaufen. Auf Basis von Berechnungen,
Überlegungen und Versuchen hat man schlussendlich die einzelnen Komponenten
erworben.

\subsubsection*{Elektronik}
Die fachlichen Ansprüche der Projektarbeit an die Elektrotechnik sind hoch.
Dies liegt insbesondere an der Tatsache, dass die geforderten Entscheidungen
und Umsetzungen des Fachbereichs in Themengebieten erfolgen, welche durch
das Studium noch nicht gedekt werden. Ebenso fehlen dort auch entsprechende
praktische Erfahrungen. Mit einer gegebenen Diskrepanz zwischen
Entscheidungsbasis und geforderten Entscheiden, wird vorausgesetzt, dass
entgegen der sonst üblichen Arbeitsweise gehandelt wird. Aus eigener
Erfahrung geht hervor, dass solch eine Arbeitsweise, welche kein
detailliertes und ausgedehntes Entwicklen von Elektronik zulässt, dazu neigt,
in eine Vielzahl unvorhergesehener Probleme zu führen. In Kombination mit
fixen Terminen und einer straffer Planung, kann dies bis hin zum Scheitern
von Systemfunktionen führen. Im Falle des Vorliegenden Projektes ist dies
fatal, dementsprechend besteht für diesen Fachbereich eine etwas angespannte
Stimmung und ein grosser Respekt gegeüber bestimmeten Aufagaben und
Implementationen.

\subsubsection*{Informatik}
Bilderkennung war komplettes Neuland. Dementsprechend ging mal mit viel
Respekt an die Aufgabe. Jedoch hat man erstaunlich einfach einen Algorithmus
durch eigene Überlegungen entwickeln können. Simple Bildoperationen wie
Zuschneiden, Kontrast verändern, Graustufen setzen und Pixel auslesen, bilden
die Grundlagen. Das ist sicherlich eine tolle Erfahrung. Durch eigene
Überlegungen und Zerlegung des Problems in Teilprobleme ein komplexe
Fragestellung lösen.

\subsubsection*{Maschinenbau}
Im Bereich der Maschinentechnik stellten die verschiedenen Berechnungen eine
Herausforderung dar. Es mussten die richtigen Annahmen getroffen werden um
plausible Resultate zu erhalten. So wurden die Wurfgeschwindigkeit,
Drehzahlen der Motoren, sowie auch der Abwurfwinkel der Vorrichtung
rechnerisch ermittelt.
Als besonders interessant blieb die Suche nach einem geeigneten
Lösungskonzept in Erinnerung. Dadurch, dass der Fantasie am Anfang keine
Grenzen gesetzt wurden, kamen allerlei sinnvolle, oder eben auch weniger
praktikable Lösungen zusammen.
