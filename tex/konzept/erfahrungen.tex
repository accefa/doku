\subsection{Erfahrungen}
Alle Mitglieder des PREN Team 39 konnten während der Erarbeitung des Konzepts wertvolle Erfahrung sammeln. Oft sind es die kleinen Dinge, welche einen grossen Lerneffekt nach sich ziehen. Sei es lediglich das Zusammenarbeiten in einem interdisziplinären Team, wobei man seinen eigenen Horizont erweitern muss, um die Probleme der anderen Disziplinen zu verstehen.\\
\\
\textbf{Generell}\\
Das erste Mal mussten die meisten Mitglieder wirtschaftlich denken. Das Budget ist nicht unbegrenzt und man kann jeweils nicht einfach die präziseste, stärkste und schnellste Komponente einkaufen. Auf Basis von Berechnungen, Überlegungen und Versuchen hat man schlussendlich die einzelnen Komponenten erworben.\\
\\
\textbf{Elektronik}\\
\\
\textbf{Informatik}\\
Bilderkennung war komplettes Neuland. Dementsprechend ging mal mit viel Respekt an die Aufgabe. Jedoch hat man erstaunlich einfach einen Algorithmus durch eigene Überlegungen entwickeln können. Simple Bildoperationen wie Zuschneiden, Kontrast verändern, Graustufen setzen und Pixel auslesen, bilden die Grundlagen. Das ist sicherlich eine tolle Erfahrung. Durch eigene Überlegungen und Zerlegung des Problems in Teilprobleme ein komplexe Fragestellung lösen.\\
\\
\textbf{Maschinenbau}\\
Im Bereich der Maschinentechnik stellten die verschiedenen Berechnungen eine Herausforderung dar. Es mussten die richtigen Annahmen getroffen werden um plausible Resultate zu erhalten. So wurden die Wurfgeschwindigkeit, Drehzahlen der Motoren, sowie auch der Abwurfwinkel der Vorrichtung rechnerisch ermittelt.
Als besonders interessant blieb die Suche nach einem geeigneten Lösungskonzept in Erinnerung. Dadurch, dass der Fantasie am Anfang keine Grenzen gesetzt wurden, kamen allerlei sinnvolle, oder eben auch weniger praktikable Lösungen zusammen.\\
\\