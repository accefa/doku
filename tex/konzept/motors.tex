\section{Motoren} \label{sec:motors}

\subsection{Schrittmotor}
Der Schrittmotor dient der Ausrichtung der Abschussvorrichtung bzw. der
Plattform, welche die Abschussvorrichtung trägt. Dieser Motor muss
daher die folgenden zwei Kriterien erfüllen.

\begin{itemize}
	\item Genauigkeit -- Der Motor richtet die Schussvorrichtung auf
		das Zielobjekt und hat somit direkten Einfluss auf die
		Genauigkeit der Vorrichtung.
	\item Kraft -- Die Kraft bestimmt die Geschwindigkeit, mit welcher
		sich die Abschussvorrichtung ausrichten kann.
\end{itemize}

Um einen Schrittmotor zu evaluieren bedarf es der Definition der oben
genannten Parameter, also der geforderten Genauigkeit als auch der
Geschwindigkeit für die Ausrichtung der Vorrichtung. Die hierfür gemachten
Annahmen und Berechnungen sind im Abschnitt \ref{sec:berechnungen}
dargelegt. Diese Berechnungen ergeben, dass der Schrittmotor ein Drehmoment
von $0.25\text{Nm} \leq M$ aufbringen muss.

\subsubsection{Mögliche Motorenmodelle}
Mit den definierten Motordaten lässt sich aus einer Vielzahl möglicher
Motoren wählen bei typischen Distributoren wie Conrad Electronics. Die
Tabelle \ref{tab:stepper-overview} bietet eine Auswahl möglicher
Schrittmotoren.

\begin{table}[h!]
	\centering
	\begin{tabular}{l l l r}
		Distributor
			& Modell
			& Moment
			& Preis \\ \hline
		Conrad
			& QSH4218-51-10-049
			& 0.49 Nm
			& 49.95 CHF \\
		Conrad
			& QSH4218-41-10-035
			& 0.35 Nm
			& 43.95 CHF \\
		Conrad
			& QSH4218-35-10-027
			& 0.27 Nm
			& 37.95 CHF
	\end{tabular}
	\caption{Übersicht möglicher Modelle von Schrittmotoren}
	\label{tab:stepper-overview}
\end{table}


\subsection{Brushlessmotor}
Ber Brushlessmotor dient der Schussabgabe und muss daher die folgenden
zwei Kriterien erfüllen.

\begin{itemize}
	\item Drehzahl -- Die Drehzahl dient der Parametrierung der
		Schussparabel und hat somit direkten Einfluss auf die
		Genauigkeit der Vorrichtung.
	\item Kraft -- Die Kraft bestimmt den Einbruch bei Belastung durch
		die Schussabgabe und die Erholzeit bis zum Wiedererreichen
		der vorgegebenen Drehzahl. Die Kraft hat somit direkten
		Einfluss auf die Kadenz der Vorrichtung.
\end{itemize}

Um einen Brushlessmotor zu evaluieren bedarf es der Definition der oben
genannten Parameter. Diese werden wiederum direkt beeinflusst vom 
mechanischen Aufbau. Hierzu gehört primär das Drehrad, welches mit seinem
Trägheitsmoment den Einbruch und somit die nötige Kraft definiert für das
Erreichen der gewünschten Kadenz. Weiter ist auch die Über- bzw.
Untersetzung ein wichtiger Faktor für die Dimensionierung des Motors, da
diese direkten Einfluss auf die Drehzahl als auch die Kraft hat.

Durch Fehlen der exakten Daten ist eine finale Evaluation derzeit nicht
möglich. Für einen groben Überblick und als Referenz für die elektrische
als auch mechanische Auslegung wird eine formale Analyse durchgeführt.

\subsubsection{Berechnungen}
Die von den Herstellern angegebenen Motordaten sind für grundlegende
Berechnungen typischerweise unzureichend, da diese für den Modellbau
vorgesehen sind. Für eine erste Dimensionierung und Berechnung der
Motordaten wird das Tool eCalc benutzt, welches ein bekanntes
Hilfsmittel ist für die Auslegung von Brushlessmotoren im Modellbau
\cite{ecalc}. Hierfür sind die Modelldaten so auslgelegt, dass der
gewählte Motor so betrieben ist, dass dieser nahe der erlaubten
Leistungsgrenze liegt. Für sämtliche Berechnungen gilt das selbe
RC-Modell mit den Standardwerten von eCalc. Individuelle Parameter
sind in den zugehörigen Tabellen dargelegt.

Aus den Berechnungen von eCalc lassen sich zwei wichtige Parameter
berechnen, welche so nicht in den Datenblättern dokumentiert sind.
Diese sind das Drehmoment $M$ und die Winkelgeschwindigkeit $\omega_m$.

\[ \omega_m = \frac{2\pi}{60} n \]
\[ M = \frac{P_m}{\omega_m} \]

\subsubsection{Simulationen}
Im Folgenden werden verschiedene die Simulationsdaten und Ergebnisse
dargelegt auf Basis des eCalc. Die hier untersuchten Motoren sind
allesamt Standardmodelle bekannter Marken aus dem Modellbau. Diese
sind bei üblichen Distributoren wie Conrad Electronics erhältlich.

\newpage
\subsubsection*{Hacker A10-12S}

\begin{table}[h!]
	\centering
	\begin{tabular}{l l l l}
		RC-Parameter & & & \\ \hline
			& Kühlung	& & Mittelmässig \\
			& Batterie	& & LiPo 10Ah, voll, 2S1P, 3.7V \\
			& Regler	& & 10A \\
			& Propeller	& & $\o=5"$, $p=2"$, $b=2$, $p_c=1.3$, $g_r=1$ \\
			& & & \\
		Resultate (max.) & & & \\ \hline
			& Motorstrom	& $I$	& 6.63 A \\
			& Motorspannung	& $U$	& 7.73 V \\
			& Drehzahl	& $n$	& 18292 min$^{-1}$ \\
			& Leistung 	& $P_e$	& 51.3 W \\
			&		& $P_m$	& 37.9 W \\
			& Wikungsgrad	& $\eta$& 73.9 \% \\
			& Motortemperatur
					& $T$	& 37 C \\
			& & & \\
		Berechnungen & & & \\ \hline
			& Winkelgeschwindigkeit
					& $\omega_m$	& 1915.5 s$^{-1}$ \\
			& Drehmoment	& $M$		& 0.0198 Nm
	\end{tabular}
	\caption{Simulationsdaten des Hacker A10-12S}
\end{table}

\begin{figure}[h!]
	\centering
	\includegraphics[width=1\textwidth]{../../fig/motor/ecalc_A10-12S.png}
	\caption{Berechnete Kennlinien des Hacker A10-12S}
	\label{fig:ecalc_A10-12S}
\end{figure}


\newpage
\subsubsection*{Hacker A20-12XL EVO}

\begin{table}[h!]
	\centering
	\begin{tabular}{l l l l}
		RC-Parameter & & & \\ \hline
			& Kühlung	& & Mittelmässig \\
			& Batterie	& & LiPo 10Ah, voll, 3S1P, 3.7V \\
			& Regler	& & 30A \\
			& Propeller	& & $\O=10"$, $p=5"$, $b=2$, $p_c=1.3$, $g_r=1$ \\
			& & & \\
		Resultate (max.) & & & \\ \hline
			& Motorstrom	& $I$	& 24.74 A \\
			& Motorspannung	& $U$	& 11.46 V \\
			& Drehzahl	& $n$	& 9676 min$^{-1}$ \\
			& Leistung 	& $P_e$	& 283.5 W \\
			&		& $P_m$	& 225.2 W \\
			& Wikungsgrad	& $\eta$& 79.4 \% \\
			& Motortemperatur
					& $T$	& 52 C \\
			& & & \\
		Berechnungen & & & \\ \hline
			& Winkelgeschwindigkeit
					& $\omega_m$	& 1013.3 s$^{-1}$ \\
			& Drehmoment	& $M$		& 0.222 Nm
	\end{tabular}
	\caption{Simulationsdaten des Hacker A20-12XL-EVO}
\end{table}

\begin{figure}[h!]
	\centering
	\includegraphics[width=1\textwidth]{../../fig/motor/ecalc_A20-12XL-EVO.png}
	\caption{Berechnete Kennlinien des Hacker A20-12XL-EVO}
	\label{fig:ecalc_A20-12XL-EVO}
\end{figure}


\newpage
\subsubsection*{Hacker A20-20L}

\begin{table}[h!]
	\centering
	\begin{tabular}{l l l l}
		RC-Parameter & & & \\ \hline
			& Kühlung	& & Mittelmässig \\
			& Batterie	& & LiPo 10Ah, voll, 3S1P, 3.7V \\
			& Regler	& & 30A \\
			& Propeller	& & $\O=10"$, $p=5"$, $b=2$, $p_c=1.3$, $g_r=1$ \\
			& & & \\
		Resultate (max.) & & & \\ \hline
			& Motorstrom	& $I$	& 21.2 A \\
			& Motorspannung	& $U$	& 11.5 V \\
			& Drehzahl	& $n$	& 9111 min$^{-1}$ \\
			& Leistung 	& $P_e$	& 243.8 W \\
			&		& $P_m$	& 186.4 W \\
			& Wikungsgrad	& $\eta$& 76.5 \% \\
			& Motortemperatur
					& $T$	& 56 C \\
			& & & \\
		Berechnungen & & & \\ \hline
			& Winkelgeschwindigkeit
					& $\omega_m$	& 954.1 s$^{-1}$ \\
			& Drehmoment	& $M$		& 0.195 Nm
	\end{tabular}
	\caption{Simulationsdaten des Hacker A20-20L}
\end{table}

\begin{figure}[h!]
	\centering
	\includegraphics[width=1\textwidth]{../../fig/motor/ecalc_A20-20L.png}
	\caption{Berechnete Kennlinien des Hacker A20-20L}
	\label{fig:ecalc_A20-20L}
\end{figure}


\newpage
\subsubsection*{Dualsky XM3548EA-5}

\begin{table}[h!]
	\centering
	\begin{tabular}{l l l l}
		RC-Parameter & & & \\ \hline
			& Kühlung	& & Mittelmässig \\
			& Batterie	& & LiPo 8Ah, voll, 3S1P, 3.7V \\
			& Regler	& & 40A \\
			& Propeller	& & $\o=13"$, $p=6"$, $b=2$, $p_c=1.3$, $g_r=1$ \\
			& & & \\
		Resultate (max.) & & & \\ \hline
			& Motorstrom	& $I$	& 37.17 A \\
			& Motorspannung	& $U$	& 11.33 V \\
			& Drehzahl	& $n$	& 7467 min$^{-1}$ \\
			& Leistung 	& $P_e$	& 421.1 W \\
			&		& $P_m$	& 359.9 W \\
			& Wikungsgrad	& $\eta$& 85.5 \% \\
			& Motortemperatur
					& $T$	& 50 C \\
			& & & \\
		Berechnungen & & & \\ \hline
			& Winkelgeschwindigkeit
					& $\omega_m$	& 781.9 s$^{-1}$ \\
			& Drehmoment	& $M$		& 0.46 Nm
	\end{tabular}
	\caption{Simulationsdaten des Dualsky XM3548EA-5}
\end{table}

\begin{figure}[h!]
	\centering
	\includegraphics[width=1\textwidth]{../../fig/motor/ecalc_XM3548EA-5.png}
	\caption{Berechnete Kennlinien des Dualsky XM3548EA-5}
	\label{fig:ecalc_xm3548ea_5}
\end{figure}

\newpage

\subsection{Gleichstrommotor}

\subsection{Encoder und Drehzahlgeber}
Die vorgesehenen Motorfunktionen verlangen lediglich beim Brushlessmotor
nach einem Feedback über die Rotation des Motors, da der Schrittmotor
definiert und feingranuliert betrieben wird und der Gleichstrommotor
keinerlei Ansprüche stellt weder an Drehzahl noch an Position.

Encoder sind relativ teuer und der Einsatz des Brushlessmotors verlangt
lediglich nach einem Feedback zur Rotation bzw. Winkelgeschwindigkeit.
Die absolute oder relative Position ist für die Anwendung nicht von
Bedeutung. Somit lässt sich ein einfaches Feedback vorsehen für die
Regelung der Drehzahl mittels optischer oder magnetischer Elemente.

Als optisches Messisnturment kann eine Lichtschranke mit 
Reflexionsstriefen oder Löchern eingesetzt werden. Diese verlangen
nur nach einer geringfügigen Modifikation des Rotierenden Körpers und
sind relativ günstig. Optische Messtechnik hat den Nachteil, dass
Störungen relativ leicht in die Messung einfliessen können, was fatale
Folgen für die Regelung hat. Magnetische Messinstrumente sind gegenüber
Störungen deutlich resistenter, da hierfür starke Magnetfelder benötigt
werden, welche so nicht einfach auftreten. Der Einsatz solcher Messtechnik
verlangt jedoch nach einer Modifikaiton der Mechanik, da Magnete in den
rotierenden Körper einebaut werden müssen. Dies birgt ein gewisses
Risiko für mechanische Unwucht des Rotationskörpers.

\subsubsection{Magnetischer Drehzahlgeber}
Um einen eigenen magnetischen Drehzahlgeber zu erstellen wird ein
sog. Hall-Effekt-Schalter eingesetzt. Dieser reagiert mit seinem Ausgang
auf ein auftretendes Magnetfeld. Das Gegenstück zum Hall-Effekt-Schalter
ist ein Magnet, welcher in das Rotierende Objekt eingebaut wird. Aus 
mechanischen Gründen, wie etwa der Unwucht, werden typisch 2 Magnete
oder ein Vielfaches davon in den rotierenden Körper eingebaut.

Bei der Rotation des Körpers entstehen durch das Vorbeigehen der Magnete
am Hall-Effekt-Schalter Pulse. Aus diesen Pulse lässt sich mittels einer
Zeitmessung direkt die Drehzahl bestimmen. Dies Abbildung 
\ref{fig:hall-effekt-schalter} illustriert das Prinzip anhand eines
Beispiels mit einem Magneten am Rotationskörper.

\begin{figure}[h!]
	\centering
	\begin{tikzpicture}
		% Koordinaten
		\draw[->] (-0.5, 0) -- (8, 0) node[anchor=north] {$t$};
		\draw[->] (0, -1.5) -- (0, 3) node[anchor=east] {$u,\varphi$};
		% Rotation
		\draw[blue] (0,0) sin (1,1) cos (2,0) sin (3,-1) cos (4,0)
			sin (5,1) cos (6,0) sin (7,-1)
			node[right] {$\varphi$};
		% Signal
		\draw[-, thick, red]
			(0,0) -- (0.8,0) -- (0.8,2) -- (1.2,2) -- (1.2,0) -- 
			(4.8,0) -- (4.8,2) -- (5.2,2) -- (5.2,0) -- (7.5,0);
		% Messung
		\draw[<->] (0.8,1.5) -- (4.8,1.5) node[midway, above] {$t_{r}$};
	\end{tikzpicture}
	\caption{Vereinfachtes Puls-Feedback eines Hall-Effekt-Scahlters}
	\label{fig:hall-effekt-schalter}
\end{figure}

Ein solches Verfahren lohnt sich bei schnellen Winkelgeschwindigkeiten
und ist für diesen Anwendungsfall sehr effizient. Zugehörige
Hall-Effekt-Schalter lassen sich einfach montieren und sind gegen Störungen
sehr robust. Ein mögliches Modell für einen Hall-Effekt-Schalter ist der
AH180N. Dieser biete einen Open-Drain Ausgang welcher somit logische Pegel
liefert (siehe Abbildung \ref{fig:AH180N_functional}).

\begin{figure}[h!]
	\centering
	\includegraphics[width=0.75\textwidth]{../../fig/motor/AH180N_functional.png}
	\caption{Funktionelles Blockschaltbild des Hall-Effekt-Scahlters AH180N}
	\label{fig:AH180N_functional}
\end{figure}

Interessant ist diese Art von Drehzahl-Geber insbesondere durch ihren
geringen Preis, denn solche Hall-Effekt-Schalter wie der AH180N befinden
sich im Preissegment < 1 CHF.
