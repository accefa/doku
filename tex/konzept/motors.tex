\section{Motoren} \label{sec:motors}

\subsection{Schrittmotor}

\subsection{Brushlessmotor}

\subsubsection{Anforderungen}
Ber Brushlessmotor dient der Schussabgabe und muss daher die folgenden
zwei Kriterien erfüllen.

\begin{description}
	\item[Drehzahl]
		Die Drehzahl dient der Parametrierung der Schussparabel
		und hat somit direkten Einfluss auf die Genauigkeit der
		Vorrichtung.
	\item[Kraft]
		Die Kraft bestimmt den Einbruch bei Belastung durch die
		Schussabgabe und die Erholzeit bis zum wiedererreichen
		der vorgegebenen Drehzahl. Die Kraft hat somit direkten
		Einfluss auf die Kadenz der Vorrichtung.
\end{description}

Um einen Brushlessmotor zu evaluieren bedarf es der Definition der oben
genannten Parameter. Diese werden wiederum direkt beeinflusst vom 
mechanischen Aufbau. Hierzu gehört primär das Drehrad, welches mit seinem
Trägheitsmoment den Einbruch und somit die nötige Kraft definiert für das
Erreichen der gewünschten Kadenz. Weiter ist auch die Über- bzw.
Untersetzung ein wichtiger Faktor für die Dimensionierung des Motors, da
dieses direkten Einfluss auf die Drehzahl als auch die Kraft hat.

Durch Fehlen der exakten Daten ist eine finale Evaluation derzeit nicht
möglich. Für einen groben Überblick und als Referenz für die elektrische
als auch mechanische Auslegung wird eine formale Analyse durchgeführt.

\subsubsection{Berechnungen}
Die von den Herstellern angegebenen Motordaten sind für grundlegende
Berechnungen typischerweise unzureichend, da diese für den Modellbau
vorgesehen sind. Für eine erste Dimensionierung und Berechnung der
Motordaten wird das Tool eCalc benutzt, welches ein bekanntes
Hilfsmittel ist für die Auslegung von Brushlessmotoren im Modellbau
\cite{ecalc}. Hierfür sind die Modelldaten so auslgelegt, dass der
gewählte Motor so betrieben ist, dass dieser nahe der erlaubten
Leistungsgrenze liegt.

\subsubsection*{Hacker A10-12S}
\begin{table}[h!]
	\centering
	\begin{tabular}{l l l l}
		RC-Parameter & & & \\ \hline
			& Kühlung	& & Mittelmässig \\
			& Batterie	& & LiPo 10Ah, voll, 2S1P, 3.7V \\
			& Regler	& & 10A \\
			& Propeller	& & $\O=5"$, $p=2"$, $b=2$, $p_c=1.3$, $g_r=1$ \\
			& & & \\
		Resultate (max.) & & & \\ \hline
			& Motorstrom	& $I$	& 6.63 A \\
			& Motorspannung	& $U$	& 7.73 V \\
			& Drehzahl	& $n$	& 18292 min$^{-1}$ \\
			& Leistung 	& $P_e$	& 51.3 W \\
			&		& $P_m$	& 37.9 W \\
			& Wikungsgrad	& $\eta$& 73.9 \% \\
			& Motortemperatur
					& $T$	& 37 C \\
			& & & \\
		Berechnungen & & & \\ \hline
			& Winkelgeschwindigkeit
					& $\omega_m$	& 1915.5 s$^{-1}$ \\
			& Drehmoment	& $M$		& 0.0198 Nm
	\end{tabular}
\end{table}

\subsubsection{Auswahl}

\subsection{Gleichstrommotor}
