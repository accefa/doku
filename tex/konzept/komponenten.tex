\subsection{Komponenten}
TODO Beschreibung der Komponenten
\subsubsection{Drehrad}

Das Drehrad ist für die Beschleunigung des Balles zuständig. Durch die Rotation des Rades wird der Tennisball auf die geforderte Geschwindigkeit gebracht. Die erreichte Flugweite hängt zu einem sehr grossen Teil von der Genauigkeit des Drehrades ab. Zudem muss das Drehrad imstande sein, geringe Durchmesseränderungen des Tennisballes auszugleichen. Dazu stehen zwei Lösungsansätze zur Verfügung. Bei der ersten Variante ist die Achse seitlich im Gestell mit Zugfedern befestigt und Das Drehrad besteht aus einem harten Material. Die Zugfedern sorgen für einen konstanten Anpressdruck auf den Ball. Diese Variante mit der beweglichen Achse hat den Nachteil, dass der Riemenspanner für den Zahnriemen anspruchsvoller wird. Bei der zweiten Variante ist die Achse fix mit dem Gestell verbunden. Um einen konstanten Anpressdruck auf den Ball zu garantieren, muss die Oberfläche des Drehrades elastisch sein. Die kann beispielsweise mit einem weichen Gummi- Modellbaurad erreicht werden. Eine weitere Möglichkeit besteht darin, ein Rad mit einer weichen Oberfläche zu versehen. Zum Beispiel mit einer Gummimatte oder Dämmmaterial (Akustisch). Zum jetzigen Zeitpunkt wird die zweite Variante bevorzugt. Die Achse ist bei beiden Varianten seitlich im Gestell gelagert. 
\subsubsection{Gestell}
Damit wir unsere Zielvorgabe von unter zwei Kilogramm erreichen können, wird das Gestell aus Holz oder eventuell aus Plexiglas hergestellt. Diese Variante hat den Vorteil, dass das Gehäuse auf der Laserschneidemaschine hergestellt werden kann. Holz oder Plexiglas hat jedoch den Nachteil, dass eventuelle Anbauten wie die Lagerung für das Drehrad in separaten Metallbuchsen erfolgen müssen. Das Gestell besteht aus vier Hauptelementen. Der Grundplatte, der Platte für den Aufbau und die beiden seitlichen Elemente des Aufbaus.