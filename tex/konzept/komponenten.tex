\subsection{Komponenten}
TODO Beschreibung der Komponenten
\subsubsection{Drehrad}

Das Drehrad ist für die Beschleunigung des Balles zuständig. Durch die Rotation des Rades wird der Tennisball auf die geforderte Geschwindigkeit gebracht. Die erreichte Flugweite hängt zu einem sehr grossen Teil von der Genauigkeit des Drehrades ab. Zudem muss das Drehrad imstande sein, geringe Durchmesseränderungen des Tennisballes auszugleichen. Dazu stehen zwei Lösungsansätze zur Verfügung. Bei der ersten Variante ist die Achse seitlich im Gestell mit Zugfedern befestigt und das Drehrad besteht aus einem harten Material. Die Zugfedern sorgen für einen konstanten Anpressdruck auf den Ball. Diese Variante mit der beweglichen Achse hat den Nachteil, dass der Riemenspanner für den Zahnriemen anspruchsvoller wird. Bei der zweiten Variante ist die Achse fix mit dem Gestell verbunden. Um einen konstanten Anpressdruck auf den Ball zu garantieren, muss die Oberfläche des Drehrades elastisch sein. Die kann beispielsweise mit einem weichen Gummi- Modellbaurad erreicht werden. Eine weitere Möglichkeit besteht darin, ein Rad mit einer weichen Oberfläche zu versehen. Zum Beispiel mit einer Gummimatte oder Dämmmaterial (Akustisch). Zum jetzigen Zeitpunkt wird die zweite Variante bevorzugt. Die Achse ist bei beiden Varianten seitlich im Gestell gelagert. 
\subsubsection{Basisaufbau}
Damit wir unsere Zielvorgabe von unter zwei Kilogramm erreichen können, wird das Gestell aus Holz oder eventuell aus Plexiglas hergestellt. Diese Variante hat den Vorteil, dass das Gehäuse auf der Laserschneidemaschine hergestellt werden kann. Holz oder Plexiglas hat jedoch den Nachteil, dass eventuelle Anbauten wie die Lagerung für das Drehrad in separaten Metallbuchsen erfolgen müssen. Das Gestell besteht aus vier Hauptelementen. Der Grundplatte, der Platte für den Aufbau und die beiden seitlichen Elemente des Aufbaus.

\subsubsection{Freedomboard}
Das Freedomboard dient der LowLevel Ansteuerung verschiedener
Hardwarekomponenten. Dies sind in erster Linie die Motortreiber.

Das Freedomboard bietet eine einfache Programmierung an für die
verschiedenen Peripherien und Busse. So reduziert sich der Softwareaufwand auf
die wesentlichen Funktionen und es können Echtzeitfunktionen
implementiert werden. Diese sind elementar für die Regelung
per Software.


\subsubsection{Raspberry Pi}
Das Raspberry Pi ist das zentrale Element des Systems. Das Pi ist nichts anderes als ein Mini-Computer. Ergänzt wird das Pi mit einem WLAN-Adapter, damit die drahtlose Kommunikation gewährleistet werden kann. \\
\\
Der Mini-Computer ist wie folgt ausgestattet (Modell B+):
\begin{itemize}
	\item CPU: 700Mhz Broadcom BCM2835
	\item RAM: 512 MB
	\item 4 USB Ports
	\item 40pin extendeded GPIO
	\item Full size HDMI
	\item Micro SD Slot (mit 8GB Micro SD)
	\item CSI Schnittstelle
\end{itemize}

Das Pi kann grundsätzlich zu viel. Die CPU und RAM sind für heutige Verhältnisse nicht gerade berauschend, jedoch genügend für unsere Anforderungen. Im Notfall können ressourcenintensive Berechnungen auf die externe Steuerungseinheit ausgelagert werden. Ein USB Port wird durch den WLAN-Adapter besetzt. Über die GPIOs wird das Freedom-Board verbunden. Die CSI-Schnittstelle dient dazu die PI Camera anzuhängen \cite{raspberri-b-plus-spec}.\\
\\
Auf der SD Karte wird das Betriebssystem Arch Linux installiert. Diese Linux-Distribution ist minimalistisch ausgestattet. Es braucht lediglich 64 MB Ram und weniger als 800 MB Speicherplatz. So können die Hardware-Ressourcen, welche begrenzt sind, für das wirklich wichtige eingesetzt werden \cite{arch-linux-system-requirements}.

\begin{figure}[h!]
	\centering
	\includegraphics[width=0.3\linewidth]{../../fig/raspberry-pi-b-plus.jpg}
	\caption{Raspberry Pi B+}
	\label{fig:raspberry-pi-b-plus}
\end{figure}


\subsubsection{Kamera}
Als Kamera für die Objekterkennung wurde das Raspberry Pi Kamera Modul ausgewählt. Diese Kamera lässt sich über die CSI\footnote{Camera Serial Interface}-Schnittstelle direkt an den Raspberry Pi anschliessen. Die Bedienung der Kamera auf dem Raspberry Pi ist durch das Kommandozeilen-Programm \verb|raspistill| sehr komfortabel. Der einzige Nachteil ist der fixe Fokus der Kamera. Die Kamera besitzt jedoch eine Weitwinkel-Linse, mit welcher alle Gegenstände die weiter als eineinhalb Meter scharf dargestellt werden. Da das Spielfeld zwei Meter lang ist passt diese Distanz optimal zu den Anforderungen. Die Kamera hat ein horizontales Sichtfeld von $53^\circ$ und ein vertikales Sichtfeld von $40^\circ$. Für eine Distanz von zwei Meter entspricht das einer Fläche von 2×1.3 Meter, welche von der Kamera aufgenommen wird. Die detaillierte Berechnung kann dem Abschnitt \ref{subsub:sichtfeld-der-kamera} entnommen werden. Neben zahlreichen Einstellmöglichkeiten, fällt vor allem die kompakte Bauweise der Kamera auf (20×25×10 mm). Abbildung \ref{fig:raspberry_pi_cam} zeigt die Kamera mit dem mitgelieferten Anschlusskabel. Durch die oben genannten Eigenschaften eignet sich diese Kamera optimal für dieses Projekt und erfüllt auch alle erwarteten Anforderungen.

\begin{figure}[h!]
\centering
\includegraphics[width=0.5\linewidth]{../../fig/raspberry_pi_cam}
\caption{Raspberry Pi Kamera Modul}
\label{fig:raspberry_pi_cam}
\end{figure}


\subsubsection{Bedienelement}
Die externe Steuerungseinheit (bzw. das Bedienelement) dient dazu um den ganzen Prozess mittels eines Start-Signals zu starten. Zum Ende des Prozesses erhält es das Stopp-Signal. Diese Kommunikationspfade müssen gemäss Anforderungen drahtlos erfolgen. \\
Ein weiterer wichtiger Punkt ist die Kalibrierung der Funktion "Ortung des Korbes". In der Umsetzung wird eine Benutzeroberfläche erstellt, welche auf der externen Steuerungseinheit betrieben wird (siehe UI-Skizze in Abschnitt \ref{ss-config-paramater-ortung-orb}).\\
Ein Notebook kann all diese Anforderungen problemlos umsetzen. Heutzutage ist das auch auf einem Smartphone oder Tablet kein Problem mehr. Es wird jedoch das Notebook aus folgenden Gründen bevorzugt:

\begin{itemize}
	\item Plattformunabhängigkeit: Die Implementation für das Notebook erfolgt in Java. Für jedes gängige Betriebssystem gibt eines JVM (Java Virtual Machine).
	\item Know-How: In der Java-Entwicklung ist ein breites Know-How innerhalb des Teams vorhanden.
	\item Bewährt: Java ist eine der meist genutzten Programmiersprachen und hat sich bewährt.
\end{itemize}

Es wird eine Rich-Client Implementation gewählt, weil dann sofort eine Plattform zur Verfügung steht, falls ressourcenintensive Berechnungen während des Prozess durchgeführt werden müssen. Gemäss Tests und Versuchen sollte dies jedoch nicht der Fall ein.





\subsubsection{Antriebe}
Für die drei Bewegungsfunktionen der Maschine ist je ein elektrischer
Motor zuständig. Diese elektrischen Motoren eignen sich aufgrund ihrer
Charakteristiken besonders für die zugewiesenen Funktionen. 

\begin{table}[h!]
	\centering
	\begin{tabular}{l l l}
		Funktion & kritische Anforderung & Motor \\
		\hline
		Ausrichtung
			& genaue Position
			& Schrittmotor \\
		Schussabgabe
			& genaue Drehzahl, Kraft
			& Brushlessmotor \\
		Ballvorschub
			& Kraft
			& Gleichstrommotor \\
	\end{tabular}
	\caption{Übersicht der eingesetzten Motorentypen}
\end{table}

Die finale Evaluation der drei Motoren ist noch nicht abgeschlossen
und ist Bestandteil der aktuellen Pendenzen. Die aktuellen
Evaluationsresultate und Berechnungen können im Anhang \ref{sec:motors}
eingesehen werden.


\subsubsection{Netzteile}
% to do

Die elektrische Energieversorgung wird mittels getrennter Speisungen erstellt,
da sowohl digitale Logik als auch Leistungselektronik vorhanden ist. Diese
haben sehr unterschiedliche Anforderungen an die Energieversorgung, welche
mittels getrennter Speisungen effektiv zu erfüllen sind.

\paragraph{Spannungsversorung digitaler Komponenten}
Die digitalen Komponenten der Vorrichtung beihnhalten primär das
Raspberry Pi und das Freedomboard. Das Freedomboard ist mittels eines USB
Kabels an das Raspberry Pi angeschlossen. Somit ist dieses durch den
USB Port des Raspberry Pi versorgt.

Das Raspberry Pi ist jedoch extern gespiesen. Hierzu wird ein
handelsübliches Steckernetzteil verwendet, welches über den Micro-USB
Anschluss des Raspberry Pi angeschlossen ist.

\paragraph{Spannungsversorgung der Leistungselektronik}
Die Spannungsversorgung der Leistungselektronischen Komponenten inklusive
der Motoren kann aktuell noch nicht definiert werden, da dies stark von den
eingesetzten Motoren abhängt. Hierbei gibt es je nach Motorentyp verschiedene
Anforderungen an die Quellen. Die Tabelle \ref{tab:power-requirement} zeigt
realistische Daten für die jeweiligen Motoren welche zum Einsatz kommen.

\begin{table}[h!]
	\centering
	\begin{tabular}{l r r r}
		Motorentyp
			& $U$ [V]
			& $I$ [A] 
			& $P$ [W] \\
		\hline
		Schrittmotor
			& 40
			& 2 
			& 80 \\
		Brushlessmotor
			& 12
			& 40
			& 480 \\
		Gleichstrommotor
			& 12
			& 5
			& 60 \\
	\end{tabular}
	\caption{Übersicht typischer Quellenanforderungen für verschiedene
		Motorentypen}
	\label{tab:power-requirement}
\end{table}

Kritisch sind die Anforderungen des Brushlessmotors, welcher hohe Ströme
fordert. Dies verhindert den Einsatz einfacher Netzteile sondern verlangt
nach speziellen Lösungen. Quellen welche solch hohen Ströme liefern können,
sind typischerweise Batterien, wenn ein kurzzeitiger Einsatz genügt für die
geplante Anwendung. Dies ist bei der automatischen Ballwurfmaschine der
Fall. Für Anwendung welche eine längere Betriebszeit verlangen, ist der
Einsatz von Leistungsstarken Netzteilen notwendig, welche typisch für die
geplante Anwendung hin entwicklet werden.

Für den Einsatz von Batterien spricht die hohe Strombelastbarkeit und die
Gewichtsreduktion, da diese nicht zum Gewicht der Wurfmaschine mitgerechnet
werden. Ebenso ist das Preis/Leistungsverhältnis besser für
hohe Ströme im Vergeich zu Netzteilen. Nachteilig ist, dass die Betriebszeit
der Wurfmaschine stark limitiert ist. Zudem erfordert der Einsatz von
Batterien auch entsprechende Ladegeräte, welche die Kostenersparnis von
Batterien direkt wieder ausgleichen.

