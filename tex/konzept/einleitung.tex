\section{Einleitung}
Automatisierte Systeme prägen die technische Entwicklung mit einer Dynamik,
wie dies einst die Dampfmaschine, der Verbrennungsmotor und nicht zuletzt der
Computer machten. Spezialisierte als auch alltägliche Tasks und 
Herausforderungen, die auch viel Geschick und Kraft benötigen, können immer
mehr durch autonome Maschinen durchgeführt werden. Dies liegt insbesondere
daran, dass moderne Technik nicht mehr nur der kaufkräftigen Industrie zu
Verfügung steht, sondern auch dem preiselastischen privaten Markt. Dass dies
heute möglich ist, hat verschiedene Gründe. Zum einen liegt dies an der
fortschreitenden technologischen Entwicklung. Ein wichtiger Katalysator für
die rasante Entwicklung ist der Paradigmenwechsel im Umgang mit Know-How.
Ein herausragendes Beispiel hierzu stellt die sogenannte 
``freie Software'' dar, welche mit berühmten Projekten wie GNU vieles
unserer heutigen technischen Wunder, wie etwa dem Internet, zu einer
Selbstverständlichkeit und Realität für Jedermann verhalf.
All diese Fortschritte führten zu einer immer mehr technisierten
Umgebung, welche die Zäune der Industrie längst überwunden hat und immer
mehr in den öffentlichen und privaten Raum Einzug hält. Dieser Wandel
verlangt nach intelligentem Design, einer optimalen Auslegung von
Ressourcen- und Energieverbrauch und nicht zuletzt nach einer
Kostenoptimierung. Gerade diese ermöglichen es dem einfachen Studenten an
moderne Technik zu gelangen und eine autonome Wurfmaschine zu
entwickeln. 

Die vorliegende Arbeit soll aufzeigen, wie die autonome Wurfmaschine
aufgebaut ist und wie diese funktioniert. Hierbei wird zunächst auf die
Schlüsselelemente eingegangen und deren Bedeutung und Zusammenhang
zum Gesamtsystem dargelegt. Weiter werden elementare Auslegungen, Annahmen
und Berechnungen dargelegt, welche sich auf die zuvor erstellten Recherchen
stützen. Ein weiterer Abschnitt der Arbeit ist der Darlegung von Arbeitsweise
und Organisation der Projektgruppe gewidmet. In einem weiteren Teil werden
durchgeführte Prüfungen und Tests zusammengefasst, welche für die Verifikation
der Schlüsselfunktionen dienen, auf welchen das Grundkonzept der Maschine
aufgebaut ist. Als Abschluss wird die Projektplanung vorgestellt, welche
einen roten Faden für die Realisation des Projektes legt und die stetige
Kontrolle des Projektfortschritts ermöglicht.

Die Arbeit konzentriert sich auf elementare Fragestellungen und grundlegende
Funktionen, deshalb wird auf Details von Subfunktionen und Subkomponenten
nicht eingegangen. Ebenso werden keinerlei konkrete Implementierungen 
beschrieben, da diese zum einen der Realisation vorbehalten sind und zum
anderen den Rahmen der Arbeit sprengen würden.
