\subsection{Ausblick}
Mit der Abgabe der Dokumentation zum PREN 1 ist Teil 1 abgeschlossen. Viele Teilfunktionen wurden mittels Prototypen getestet. Sowohl mechanische Aspekte wie das Drehrad für die Beschleunigung der Bälle als auch softwaretechnische Komponente wie der Algorithmus für die Ortung des Korbes. Aus diesen Gründen blicken wir mit einem gutem Bauchgefühl zuversichtlich in die Zukunft - ins PREN 2. \\
Wir sind uns bewusst, dass Theorie und Praxis auseinanderliegen können und dass wir für solche Differenzen gewappnet sein müssen. Damit das primäre Ziel im nachfolgenden Modul, die erfolgreiche Umsetzung des Konzeptes, erreicht werden kann, wird auf folgende Punkte besonders geachtet:

\begin{itemize}
	\item Die \textbf{Ortung des Korbes} ist Dreh- und Angelpunkt. Natürlich sind alle Funktionen wichtig, falls jedoch bereits diese Funktion fehlschlägt, nutzt auch ein guter Wurfmechanismus nichts. Da Know-How und Erfahrung in diesem Bereich gering sind, will man bereits möglichst früh im PREN 2 einen funktionsfähigen Algorithmus präsentieren können.
	
	\item Bei der \textbf{Beschleunigung der Bälle} ist man besorgt um die Konstanz und Präzision. Wie reproduzierbar sind feine Einstellungen bzw. kleine Unterschiede bezüglich der Umlaufsgeschwindigkeit des Rades? Zum anderen beschäftigt uns auch die Frage des Rückstosses. Hält die Maschine der Kraft stand ohne sich zu verschieben?
	
	\item Als kritischen Punkt sieht man in der \textbf{Elektronik} die Dimensionierung an. Falls der Motor die Kraft nicht aufbringt um das Drehrad auf die entsprechende Geschwindigkeit zu bringen, kann das Projekt scheitern. Das Budget ist begrenzt und die Motoren nehmen einen gewichtigen Anteil an den Kosten.
	
	\item Generell gilt es die \textbf{Planung} im Auge zu behalten. Falsch gesetzte Termine können Probleme und Stress verursachen und so die Qualität des Produktes negativ beeinflussen.
\end{itemize}