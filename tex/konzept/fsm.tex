Der allgemeine Ablauf der autonomen Wurfmaschine kann als endlicher Automat
moddeliert werden, denn es gibt einen definierten Einstiegs- und 
Ausstiegspunkt. Der Einstieg (Start) ist definiert durch das Aufsuchen des
Zielobjektes und der Ausstiegspunkt (Ende) ist gegeben durch das Fehlen
weiterer Wurfobjekte. Zwischen Start und Ende muss das Zielobjekt
identifiziert, die Maschine danach ausgerichtet, die Wurfautomatik
parametriert und schlussendlich der Wurf durchgeführt werden.

\begin{figure}[h!]
	\centering
\begin{tikzpicture}[shorten >=2pt,node distance=3cm,on grid,auto]
	\node[state,initial] 	(q_0) 				{$q_0$};
	\node[state] 		(q_1) [above right=of q_0]	{$q_1$};
	\node[state]		(q_2) [right=of q_1]		{$q_2$};
	\node[state,accepting]	(q_3) [below right=of q_2]	{$q_3$};
	\path[->]
		(q_0)	edge [loop below] node	{suche Korb}	(q_0)
		(q_0)	edge node 				{Korb entdeckt} (q_1)
		(q_1)	edge [loop above] node	{!ausgerichtet}	(q_1)
		(q_1)	edge node 				{ausgerichtet}	(q_2)
		(q_2)	edge [loop above] node	{!Solldrehzahl}	(q_2)
		(q_2)	edge node				{Solldrehzahl}	(q_3)
		(q_3)	edge [loop right] node {ausgeschossen}	(q_3)
		(q_3)	edge node				{restart}	(q_0);
\end{tikzpicture}
	\caption{Autonome Wurfmaschine modellitert als endlicher Automat}
\end{figure}
