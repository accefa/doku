\begin{figure}[h!]
	\centering
\begin{tikzpicture}[shorten >=2pt,node distance=4cm,on grid,auto]
	\node[state,initial] 	(q_0) 						{$q_0$};
	\node[state] 			(q_1) [above=of q_0]		{$q_1$};
	\node[state]			(q_2) [right=of q_1]		{$q_2$};
	\node[state]			(q_3) [right=of q_2]		{$q_3$};
	\node[state]			(q_4) [right=of q_3]		{$q_4$};
	\node[state,accepting]	(q_5) [below=of q_4]		{$q_5$};
	\path[->]
		(q_0)	edge node	{\begin{tabular}{c} Startsignal \end{tabular}} (q_1)
		(q_1)	edge node	{Korb entdeckt}	(q_2)
		(q_2)	edge node	{\begin{tabular}{c} Plattform \\ ausgerichtet \end{tabular}}	(q_3)
		(q_3)	edge node	{Drehzahl erreicht}	(q_4)
		(q_4)	edge node	{\begin{tabular}{c} Keine Bälle \\ vorhanden \end{tabular}}	(q_5);
\end{tikzpicture}
	\caption{Autonome Wurfmaschine modelliert als endlicher Automat}
\end{figure}
