\section{Funktionsübersicht}
In dieser Übersicht werden alle erarbeiteten Funktionen aufgelistet. Dazu werden konkrete Umsetzungsmöglichkeiten genannt, welche die Funktion umsetzen bzw. unterstützen könnten.\\

\begin{longtable}{l p{4cm} p{4.5cm} p{7cm}}
	\textbf{Nr} & \textbf{Funktion} & \textbf{Ideen} & \textbf{Beschreibung} \\ \hline \\ \endhead 
	1  & Energieversorgung & A Elektrisch (Netz) & Elektrizität mit dem Stromnetz als Quelle. \\
	 &  & B Elektrisch (Akku) & Elektrizität mit einem Akku als Quelle. \\
	 &  & C Pneumatisch (direkt) &  Luftdruck mit dem Druckluftnetz als Quelle. \\
	 &  & D Pneumatisch (Drucktank) & Luftruck mit einem Drucktank als Quelle. \\
	 &  & E Dampf & Dampf als Energiequelle. \\ \\
	\hline \\
	2 & Ball-Lagerung & A Magazin & Ein Magazin als Lagerung. \\
	 &  & B Korb & Ein Korb als Lagerung. \\
	 &  & C Netz & Ein Netz als Lagerung. \\
	 &  & D Rohr & Ein Rohr als Lagerung. \\ \\
	\hline \\
	3 & Kommunikation & A Handy & Ein Handy als externer Kommunikationspartner. \\
	 &  & B Laptop & Ein Laptop als externer Kommunikationspartner. \\
	 &  & C Fernbedienung & Eine Fernbedienung als externer Kommunikationspartner. \\
	 &  & D Akustisches Signal & Datenübertragung per Akustik. \\
	 &  & E Lichtsignal & Datenübertragung per Licht. \\ \\
	\hline \\
	4 & Ortung des Korbs & A Ultraschall & Mittels Ultraschall den Ort des Korbs detektieren. \\
	 &  & B Laser & Mittels Laser den Ort des Korbs detektieren. \\
	 &  & C Optik & Mittels Optik den Ort des Korbs detektieren (Kamera). \\
	 &  & D Wärmebild & Mittels Wärmebild den Ort des Korbs detektieren. \\
	 &  & E Radar & Mittels Radar den Ort des Korbs detektieren. \\ \\
	\hline \\
	5 & Maschinen Positionierung & A Fix & Maschine fixiert an einem Ort. \\
	 &  & B Spring in Korb & Maschine springt komplett in den Korb. \\
	 &  & C Fährt & Maschine fährt an einen Ort um sich zu positionieren. \\
	 &  & D Rollt & Maschine rollt an einen Ort um sich zu positionieren. \\
	 &  & E Fliegt & Maschine fliegt. \\ \\
	\hline \\
	6 & Transport der Bälle & A Drehräder (Reibung) & Die Bälle werden durch die Reibung an den Drehräder beschleunigt. \\
	 &  & B Drehräder (Formschlüssig) & Die Bälle gewinnen an Geschwindigkeit durch die formschlüssigen Elemente an den Drehrädern.  \\
	 &  & C Katapult & Die Bälle werden katapultiert. \\
	 &  & D Ausfahrbarer Zylinder & Ein Zylinder, welcher einen Stossimpuls gibt.  \\
	 &  & E Fallbeschleunigung & Die Bälle fliegen aus der Höhe in die Tiefe. \\
	 &  & F Feder & Mit Federkraft die Bälle Beschleunigung.  \\
	 &  & G Luft & Die Bälle mit Luft beschleunigen.  \\ \\
	\hline \\ 
	7 & Computer & A Bordcomputer & Steuereinheit der Maschine. \\ \\
	\hline \\
	8 & Maschinen Ausrichtung & A Vertikale Ausrichtung & Maschine richtet sich selbst vertikal aus. \\
	 &  & B Horizontale Ausrichtung & Maschine richtet sich horizontal aus. \\
 	\caption{Funktionsübersicht}
 	\label{tab:quelle}
\end{longtable}

